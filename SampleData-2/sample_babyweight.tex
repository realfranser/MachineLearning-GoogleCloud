\documentclass[11pt]{article}

    \usepackage[breakable]{tcolorbox}
    \usepackage{parskip} % Stop auto-indenting (to mimic markdown behaviour)
    
    \usepackage{iftex}
    \ifPDFTeX
    	\usepackage[T1]{fontenc}
    	\usepackage{mathpazo}
    \else
    	\usepackage{fontspec}
    \fi

    % Basic figure setup, for now with no caption control since it's done
    % automatically by Pandoc (which extracts ![](path) syntax from Markdown).
    \usepackage{graphicx}
    % Maintain compatibility with old templates. Remove in nbconvert 6.0
    \let\Oldincludegraphics\includegraphics
    % Ensure that by default, figures have no caption (until we provide a
    % proper Figure object with a Caption API and a way to capture that
    % in the conversion process - todo).
    \usepackage{caption}
    \DeclareCaptionFormat{nocaption}{}
    \captionsetup{format=nocaption,aboveskip=0pt,belowskip=0pt}

    \usepackage[Export]{adjustbox} % Used to constrain images to a maximum size
    \adjustboxset{max size={0.9\linewidth}{0.9\paperheight}}
    \usepackage{float}
    \floatplacement{figure}{H} % forces figures to be placed at the correct location
    \usepackage{xcolor} % Allow colors to be defined
    \usepackage{enumerate} % Needed for markdown enumerations to work
    \usepackage{geometry} % Used to adjust the document margins
    \usepackage{amsmath} % Equations
    \usepackage{amssymb} % Equations
    \usepackage{textcomp} % defines textquotesingle
    % Hack from http://tex.stackexchange.com/a/47451/13684:
    \AtBeginDocument{%
        \def\PYZsq{\textquotesingle}% Upright quotes in Pygmentized code
    }
    \usepackage{upquote} % Upright quotes for verbatim code
    \usepackage{eurosym} % defines \euro
    \usepackage[mathletters]{ucs} % Extended unicode (utf-8) support
    \usepackage{fancyvrb} % verbatim replacement that allows latex
    \usepackage{grffile} % extends the file name processing of package graphics 
                         % to support a larger range
    \makeatletter % fix for grffile with XeLaTeX
    \def\Gread@@xetex#1{%
      \IfFileExists{"\Gin@base".bb}%
      {\Gread@eps{\Gin@base.bb}}%
      {\Gread@@xetex@aux#1}%
    }
    \makeatother

    % The hyperref package gives us a pdf with properly built
    % internal navigation ('pdf bookmarks' for the table of contents,
    % internal cross-reference links, web links for URLs, etc.)
    \usepackage{hyperref}
    % The default LaTeX title has an obnoxious amount of whitespace. By default,
    % titling removes some of it. It also provides customization options.
    \usepackage{titling}
    \usepackage{longtable} % longtable support required by pandoc >1.10
    \usepackage{booktabs}  % table support for pandoc > 1.12.2
    \usepackage[inline]{enumitem} % IRkernel/repr support (it uses the enumerate* environment)
    \usepackage[normalem]{ulem} % ulem is needed to support strikethroughs (\sout)
                                % normalem makes italics be italics, not underlines
    \usepackage{mathrsfs}
    

    
    % Colors for the hyperref package
    \definecolor{urlcolor}{rgb}{0,.145,.698}
    \definecolor{linkcolor}{rgb}{.71,0.21,0.01}
    \definecolor{citecolor}{rgb}{.12,.54,.11}

    % ANSI colors
    \definecolor{ansi-black}{HTML}{3E424D}
    \definecolor{ansi-black-intense}{HTML}{282C36}
    \definecolor{ansi-red}{HTML}{E75C58}
    \definecolor{ansi-red-intense}{HTML}{B22B31}
    \definecolor{ansi-green}{HTML}{00A250}
    \definecolor{ansi-green-intense}{HTML}{007427}
    \definecolor{ansi-yellow}{HTML}{DDB62B}
    \definecolor{ansi-yellow-intense}{HTML}{B27D12}
    \definecolor{ansi-blue}{HTML}{208FFB}
    \definecolor{ansi-blue-intense}{HTML}{0065CA}
    \definecolor{ansi-magenta}{HTML}{D160C4}
    \definecolor{ansi-magenta-intense}{HTML}{A03196}
    \definecolor{ansi-cyan}{HTML}{60C6C8}
    \definecolor{ansi-cyan-intense}{HTML}{258F8F}
    \definecolor{ansi-white}{HTML}{C5C1B4}
    \definecolor{ansi-white-intense}{HTML}{A1A6B2}
    \definecolor{ansi-default-inverse-fg}{HTML}{FFFFFF}
    \definecolor{ansi-default-inverse-bg}{HTML}{000000}

    % commands and environments needed by pandoc snippets
    % extracted from the output of `pandoc -s`
    \providecommand{\tightlist}{%
      \setlength{\itemsep}{0pt}\setlength{\parskip}{0pt}}
    \DefineVerbatimEnvironment{Highlighting}{Verbatim}{commandchars=\\\{\}}
    % Add ',fontsize=\small' for more characters per line
    \newenvironment{Shaded}{}{}
    \newcommand{\KeywordTok}[1]{\textcolor[rgb]{0.00,0.44,0.13}{\textbf{{#1}}}}
    \newcommand{\DataTypeTok}[1]{\textcolor[rgb]{0.56,0.13,0.00}{{#1}}}
    \newcommand{\DecValTok}[1]{\textcolor[rgb]{0.25,0.63,0.44}{{#1}}}
    \newcommand{\BaseNTok}[1]{\textcolor[rgb]{0.25,0.63,0.44}{{#1}}}
    \newcommand{\FloatTok}[1]{\textcolor[rgb]{0.25,0.63,0.44}{{#1}}}
    \newcommand{\CharTok}[1]{\textcolor[rgb]{0.25,0.44,0.63}{{#1}}}
    \newcommand{\StringTok}[1]{\textcolor[rgb]{0.25,0.44,0.63}{{#1}}}
    \newcommand{\CommentTok}[1]{\textcolor[rgb]{0.38,0.63,0.69}{\textit{{#1}}}}
    \newcommand{\OtherTok}[1]{\textcolor[rgb]{0.00,0.44,0.13}{{#1}}}
    \newcommand{\AlertTok}[1]{\textcolor[rgb]{1.00,0.00,0.00}{\textbf{{#1}}}}
    \newcommand{\FunctionTok}[1]{\textcolor[rgb]{0.02,0.16,0.49}{{#1}}}
    \newcommand{\RegionMarkerTok}[1]{{#1}}
    \newcommand{\ErrorTok}[1]{\textcolor[rgb]{1.00,0.00,0.00}{\textbf{{#1}}}}
    \newcommand{\NormalTok}[1]{{#1}}
    
    % Additional commands for more recent versions of Pandoc
    \newcommand{\ConstantTok}[1]{\textcolor[rgb]{0.53,0.00,0.00}{{#1}}}
    \newcommand{\SpecialCharTok}[1]{\textcolor[rgb]{0.25,0.44,0.63}{{#1}}}
    \newcommand{\VerbatimStringTok}[1]{\textcolor[rgb]{0.25,0.44,0.63}{{#1}}}
    \newcommand{\SpecialStringTok}[1]{\textcolor[rgb]{0.73,0.40,0.53}{{#1}}}
    \newcommand{\ImportTok}[1]{{#1}}
    \newcommand{\DocumentationTok}[1]{\textcolor[rgb]{0.73,0.13,0.13}{\textit{{#1}}}}
    \newcommand{\AnnotationTok}[1]{\textcolor[rgb]{0.38,0.63,0.69}{\textbf{\textit{{#1}}}}}
    \newcommand{\CommentVarTok}[1]{\textcolor[rgb]{0.38,0.63,0.69}{\textbf{\textit{{#1}}}}}
    \newcommand{\VariableTok}[1]{\textcolor[rgb]{0.10,0.09,0.49}{{#1}}}
    \newcommand{\ControlFlowTok}[1]{\textcolor[rgb]{0.00,0.44,0.13}{\textbf{{#1}}}}
    \newcommand{\OperatorTok}[1]{\textcolor[rgb]{0.40,0.40,0.40}{{#1}}}
    \newcommand{\BuiltInTok}[1]{{#1}}
    \newcommand{\ExtensionTok}[1]{{#1}}
    \newcommand{\PreprocessorTok}[1]{\textcolor[rgb]{0.74,0.48,0.00}{{#1}}}
    \newcommand{\AttributeTok}[1]{\textcolor[rgb]{0.49,0.56,0.16}{{#1}}}
    \newcommand{\InformationTok}[1]{\textcolor[rgb]{0.38,0.63,0.69}{\textbf{\textit{{#1}}}}}
    \newcommand{\WarningTok}[1]{\textcolor[rgb]{0.38,0.63,0.69}{\textbf{\textit{{#1}}}}}
    
    
    % Define a nice break command that doesn't care if a line doesn't already
    % exist.
    \def\br{\hspace*{\fill} \\* }
    % Math Jax compatibility definitions
    \def\gt{>}
    \def\lt{<}
    \let\Oldtex\TeX
    \let\Oldlatex\LaTeX
    \renewcommand{\TeX}{\textrm{\Oldtex}}
    \renewcommand{\LaTeX}{\textrm{\Oldlatex}}
    % Document parameters
    % Document title
    \title{sample\_babyweight}
    
    
    
    
    
% Pygments definitions
\makeatletter
\def\PY@reset{\let\PY@it=\relax \let\PY@bf=\relax%
    \let\PY@ul=\relax \let\PY@tc=\relax%
    \let\PY@bc=\relax \let\PY@ff=\relax}
\def\PY@tok#1{\csname PY@tok@#1\endcsname}
\def\PY@toks#1+{\ifx\relax#1\empty\else%
    \PY@tok{#1}\expandafter\PY@toks\fi}
\def\PY@do#1{\PY@bc{\PY@tc{\PY@ul{%
    \PY@it{\PY@bf{\PY@ff{#1}}}}}}}
\def\PY#1#2{\PY@reset\PY@toks#1+\relax+\PY@do{#2}}

\expandafter\def\csname PY@tok@w\endcsname{\def\PY@tc##1{\textcolor[rgb]{0.73,0.73,0.73}{##1}}}
\expandafter\def\csname PY@tok@c\endcsname{\let\PY@it=\textit\def\PY@tc##1{\textcolor[rgb]{0.25,0.50,0.50}{##1}}}
\expandafter\def\csname PY@tok@cp\endcsname{\def\PY@tc##1{\textcolor[rgb]{0.74,0.48,0.00}{##1}}}
\expandafter\def\csname PY@tok@k\endcsname{\let\PY@bf=\textbf\def\PY@tc##1{\textcolor[rgb]{0.00,0.50,0.00}{##1}}}
\expandafter\def\csname PY@tok@kp\endcsname{\def\PY@tc##1{\textcolor[rgb]{0.00,0.50,0.00}{##1}}}
\expandafter\def\csname PY@tok@kt\endcsname{\def\PY@tc##1{\textcolor[rgb]{0.69,0.00,0.25}{##1}}}
\expandafter\def\csname PY@tok@o\endcsname{\def\PY@tc##1{\textcolor[rgb]{0.40,0.40,0.40}{##1}}}
\expandafter\def\csname PY@tok@ow\endcsname{\let\PY@bf=\textbf\def\PY@tc##1{\textcolor[rgb]{0.67,0.13,1.00}{##1}}}
\expandafter\def\csname PY@tok@nb\endcsname{\def\PY@tc##1{\textcolor[rgb]{0.00,0.50,0.00}{##1}}}
\expandafter\def\csname PY@tok@nf\endcsname{\def\PY@tc##1{\textcolor[rgb]{0.00,0.00,1.00}{##1}}}
\expandafter\def\csname PY@tok@nc\endcsname{\let\PY@bf=\textbf\def\PY@tc##1{\textcolor[rgb]{0.00,0.00,1.00}{##1}}}
\expandafter\def\csname PY@tok@nn\endcsname{\let\PY@bf=\textbf\def\PY@tc##1{\textcolor[rgb]{0.00,0.00,1.00}{##1}}}
\expandafter\def\csname PY@tok@ne\endcsname{\let\PY@bf=\textbf\def\PY@tc##1{\textcolor[rgb]{0.82,0.25,0.23}{##1}}}
\expandafter\def\csname PY@tok@nv\endcsname{\def\PY@tc##1{\textcolor[rgb]{0.10,0.09,0.49}{##1}}}
\expandafter\def\csname PY@tok@no\endcsname{\def\PY@tc##1{\textcolor[rgb]{0.53,0.00,0.00}{##1}}}
\expandafter\def\csname PY@tok@nl\endcsname{\def\PY@tc##1{\textcolor[rgb]{0.63,0.63,0.00}{##1}}}
\expandafter\def\csname PY@tok@ni\endcsname{\let\PY@bf=\textbf\def\PY@tc##1{\textcolor[rgb]{0.60,0.60,0.60}{##1}}}
\expandafter\def\csname PY@tok@na\endcsname{\def\PY@tc##1{\textcolor[rgb]{0.49,0.56,0.16}{##1}}}
\expandafter\def\csname PY@tok@nt\endcsname{\let\PY@bf=\textbf\def\PY@tc##1{\textcolor[rgb]{0.00,0.50,0.00}{##1}}}
\expandafter\def\csname PY@tok@nd\endcsname{\def\PY@tc##1{\textcolor[rgb]{0.67,0.13,1.00}{##1}}}
\expandafter\def\csname PY@tok@s\endcsname{\def\PY@tc##1{\textcolor[rgb]{0.73,0.13,0.13}{##1}}}
\expandafter\def\csname PY@tok@sd\endcsname{\let\PY@it=\textit\def\PY@tc##1{\textcolor[rgb]{0.73,0.13,0.13}{##1}}}
\expandafter\def\csname PY@tok@si\endcsname{\let\PY@bf=\textbf\def\PY@tc##1{\textcolor[rgb]{0.73,0.40,0.53}{##1}}}
\expandafter\def\csname PY@tok@se\endcsname{\let\PY@bf=\textbf\def\PY@tc##1{\textcolor[rgb]{0.73,0.40,0.13}{##1}}}
\expandafter\def\csname PY@tok@sr\endcsname{\def\PY@tc##1{\textcolor[rgb]{0.73,0.40,0.53}{##1}}}
\expandafter\def\csname PY@tok@ss\endcsname{\def\PY@tc##1{\textcolor[rgb]{0.10,0.09,0.49}{##1}}}
\expandafter\def\csname PY@tok@sx\endcsname{\def\PY@tc##1{\textcolor[rgb]{0.00,0.50,0.00}{##1}}}
\expandafter\def\csname PY@tok@m\endcsname{\def\PY@tc##1{\textcolor[rgb]{0.40,0.40,0.40}{##1}}}
\expandafter\def\csname PY@tok@gh\endcsname{\let\PY@bf=\textbf\def\PY@tc##1{\textcolor[rgb]{0.00,0.00,0.50}{##1}}}
\expandafter\def\csname PY@tok@gu\endcsname{\let\PY@bf=\textbf\def\PY@tc##1{\textcolor[rgb]{0.50,0.00,0.50}{##1}}}
\expandafter\def\csname PY@tok@gd\endcsname{\def\PY@tc##1{\textcolor[rgb]{0.63,0.00,0.00}{##1}}}
\expandafter\def\csname PY@tok@gi\endcsname{\def\PY@tc##1{\textcolor[rgb]{0.00,0.63,0.00}{##1}}}
\expandafter\def\csname PY@tok@gr\endcsname{\def\PY@tc##1{\textcolor[rgb]{1.00,0.00,0.00}{##1}}}
\expandafter\def\csname PY@tok@ge\endcsname{\let\PY@it=\textit}
\expandafter\def\csname PY@tok@gs\endcsname{\let\PY@bf=\textbf}
\expandafter\def\csname PY@tok@gp\endcsname{\let\PY@bf=\textbf\def\PY@tc##1{\textcolor[rgb]{0.00,0.00,0.50}{##1}}}
\expandafter\def\csname PY@tok@go\endcsname{\def\PY@tc##1{\textcolor[rgb]{0.53,0.53,0.53}{##1}}}
\expandafter\def\csname PY@tok@gt\endcsname{\def\PY@tc##1{\textcolor[rgb]{0.00,0.27,0.87}{##1}}}
\expandafter\def\csname PY@tok@err\endcsname{\def\PY@bc##1{\setlength{\fboxsep}{0pt}\fcolorbox[rgb]{1.00,0.00,0.00}{1,1,1}{\strut ##1}}}
\expandafter\def\csname PY@tok@kc\endcsname{\let\PY@bf=\textbf\def\PY@tc##1{\textcolor[rgb]{0.00,0.50,0.00}{##1}}}
\expandafter\def\csname PY@tok@kd\endcsname{\let\PY@bf=\textbf\def\PY@tc##1{\textcolor[rgb]{0.00,0.50,0.00}{##1}}}
\expandafter\def\csname PY@tok@kn\endcsname{\let\PY@bf=\textbf\def\PY@tc##1{\textcolor[rgb]{0.00,0.50,0.00}{##1}}}
\expandafter\def\csname PY@tok@kr\endcsname{\let\PY@bf=\textbf\def\PY@tc##1{\textcolor[rgb]{0.00,0.50,0.00}{##1}}}
\expandafter\def\csname PY@tok@bp\endcsname{\def\PY@tc##1{\textcolor[rgb]{0.00,0.50,0.00}{##1}}}
\expandafter\def\csname PY@tok@fm\endcsname{\def\PY@tc##1{\textcolor[rgb]{0.00,0.00,1.00}{##1}}}
\expandafter\def\csname PY@tok@vc\endcsname{\def\PY@tc##1{\textcolor[rgb]{0.10,0.09,0.49}{##1}}}
\expandafter\def\csname PY@tok@vg\endcsname{\def\PY@tc##1{\textcolor[rgb]{0.10,0.09,0.49}{##1}}}
\expandafter\def\csname PY@tok@vi\endcsname{\def\PY@tc##1{\textcolor[rgb]{0.10,0.09,0.49}{##1}}}
\expandafter\def\csname PY@tok@vm\endcsname{\def\PY@tc##1{\textcolor[rgb]{0.10,0.09,0.49}{##1}}}
\expandafter\def\csname PY@tok@sa\endcsname{\def\PY@tc##1{\textcolor[rgb]{0.73,0.13,0.13}{##1}}}
\expandafter\def\csname PY@tok@sb\endcsname{\def\PY@tc##1{\textcolor[rgb]{0.73,0.13,0.13}{##1}}}
\expandafter\def\csname PY@tok@sc\endcsname{\def\PY@tc##1{\textcolor[rgb]{0.73,0.13,0.13}{##1}}}
\expandafter\def\csname PY@tok@dl\endcsname{\def\PY@tc##1{\textcolor[rgb]{0.73,0.13,0.13}{##1}}}
\expandafter\def\csname PY@tok@s2\endcsname{\def\PY@tc##1{\textcolor[rgb]{0.73,0.13,0.13}{##1}}}
\expandafter\def\csname PY@tok@sh\endcsname{\def\PY@tc##1{\textcolor[rgb]{0.73,0.13,0.13}{##1}}}
\expandafter\def\csname PY@tok@s1\endcsname{\def\PY@tc##1{\textcolor[rgb]{0.73,0.13,0.13}{##1}}}
\expandafter\def\csname PY@tok@mb\endcsname{\def\PY@tc##1{\textcolor[rgb]{0.40,0.40,0.40}{##1}}}
\expandafter\def\csname PY@tok@mf\endcsname{\def\PY@tc##1{\textcolor[rgb]{0.40,0.40,0.40}{##1}}}
\expandafter\def\csname PY@tok@mh\endcsname{\def\PY@tc##1{\textcolor[rgb]{0.40,0.40,0.40}{##1}}}
\expandafter\def\csname PY@tok@mi\endcsname{\def\PY@tc##1{\textcolor[rgb]{0.40,0.40,0.40}{##1}}}
\expandafter\def\csname PY@tok@il\endcsname{\def\PY@tc##1{\textcolor[rgb]{0.40,0.40,0.40}{##1}}}
\expandafter\def\csname PY@tok@mo\endcsname{\def\PY@tc##1{\textcolor[rgb]{0.40,0.40,0.40}{##1}}}
\expandafter\def\csname PY@tok@ch\endcsname{\let\PY@it=\textit\def\PY@tc##1{\textcolor[rgb]{0.25,0.50,0.50}{##1}}}
\expandafter\def\csname PY@tok@cm\endcsname{\let\PY@it=\textit\def\PY@tc##1{\textcolor[rgb]{0.25,0.50,0.50}{##1}}}
\expandafter\def\csname PY@tok@cpf\endcsname{\let\PY@it=\textit\def\PY@tc##1{\textcolor[rgb]{0.25,0.50,0.50}{##1}}}
\expandafter\def\csname PY@tok@c1\endcsname{\let\PY@it=\textit\def\PY@tc##1{\textcolor[rgb]{0.25,0.50,0.50}{##1}}}
\expandafter\def\csname PY@tok@cs\endcsname{\let\PY@it=\textit\def\PY@tc##1{\textcolor[rgb]{0.25,0.50,0.50}{##1}}}

\def\PYZbs{\char`\\}
\def\PYZus{\char`\_}
\def\PYZob{\char`\{}
\def\PYZcb{\char`\}}
\def\PYZca{\char`\^}
\def\PYZam{\char`\&}
\def\PYZlt{\char`\<}
\def\PYZgt{\char`\>}
\def\PYZsh{\char`\#}
\def\PYZpc{\char`\%}
\def\PYZdl{\char`\$}
\def\PYZhy{\char`\-}
\def\PYZsq{\char`\'}
\def\PYZdq{\char`\"}
\def\PYZti{\char`\~}
% for compatibility with earlier versions
\def\PYZat{@}
\def\PYZlb{[}
\def\PYZrb{]}
\makeatother


    % For linebreaks inside Verbatim environment from package fancyvrb. 
    \makeatletter
        \newbox\Wrappedcontinuationbox 
        \newbox\Wrappedvisiblespacebox 
        \newcommand*\Wrappedvisiblespace {\textcolor{red}{\textvisiblespace}} 
        \newcommand*\Wrappedcontinuationsymbol {\textcolor{red}{\llap{\tiny$\m@th\hookrightarrow$}}} 
        \newcommand*\Wrappedcontinuationindent {3ex } 
        \newcommand*\Wrappedafterbreak {\kern\Wrappedcontinuationindent\copy\Wrappedcontinuationbox} 
        % Take advantage of the already applied Pygments mark-up to insert 
        % potential linebreaks for TeX processing. 
        %        {, <, #, %, $, ' and ": go to next line. 
        %        _, }, ^, &, >, - and ~: stay at end of broken line. 
        % Use of \textquotesingle for straight quote. 
        \newcommand*\Wrappedbreaksatspecials {% 
            \def\PYGZus{\discretionary{\char`\_}{\Wrappedafterbreak}{\char`\_}}% 
            \def\PYGZob{\discretionary{}{\Wrappedafterbreak\char`\{}{\char`\{}}% 
            \def\PYGZcb{\discretionary{\char`\}}{\Wrappedafterbreak}{\char`\}}}% 
            \def\PYGZca{\discretionary{\char`\^}{\Wrappedafterbreak}{\char`\^}}% 
            \def\PYGZam{\discretionary{\char`\&}{\Wrappedafterbreak}{\char`\&}}% 
            \def\PYGZlt{\discretionary{}{\Wrappedafterbreak\char`\<}{\char`\<}}% 
            \def\PYGZgt{\discretionary{\char`\>}{\Wrappedafterbreak}{\char`\>}}% 
            \def\PYGZsh{\discretionary{}{\Wrappedafterbreak\char`\#}{\char`\#}}% 
            \def\PYGZpc{\discretionary{}{\Wrappedafterbreak\char`\%}{\char`\%}}% 
            \def\PYGZdl{\discretionary{}{\Wrappedafterbreak\char`\$}{\char`\$}}% 
            \def\PYGZhy{\discretionary{\char`\-}{\Wrappedafterbreak}{\char`\-}}% 
            \def\PYGZsq{\discretionary{}{\Wrappedafterbreak\textquotesingle}{\textquotesingle}}% 
            \def\PYGZdq{\discretionary{}{\Wrappedafterbreak\char`\"}{\char`\"}}% 
            \def\PYGZti{\discretionary{\char`\~}{\Wrappedafterbreak}{\char`\~}}% 
        } 
        % Some characters . , ; ? ! / are not pygmentized. 
        % This macro makes them "active" and they will insert potential linebreaks 
        \newcommand*\Wrappedbreaksatpunct {% 
            \lccode`\~`\.\lowercase{\def~}{\discretionary{\hbox{\char`\.}}{\Wrappedafterbreak}{\hbox{\char`\.}}}% 
            \lccode`\~`\,\lowercase{\def~}{\discretionary{\hbox{\char`\,}}{\Wrappedafterbreak}{\hbox{\char`\,}}}% 
            \lccode`\~`\;\lowercase{\def~}{\discretionary{\hbox{\char`\;}}{\Wrappedafterbreak}{\hbox{\char`\;}}}% 
            \lccode`\~`\:\lowercase{\def~}{\discretionary{\hbox{\char`\:}}{\Wrappedafterbreak}{\hbox{\char`\:}}}% 
            \lccode`\~`\?\lowercase{\def~}{\discretionary{\hbox{\char`\?}}{\Wrappedafterbreak}{\hbox{\char`\?}}}% 
            \lccode`\~`\!\lowercase{\def~}{\discretionary{\hbox{\char`\!}}{\Wrappedafterbreak}{\hbox{\char`\!}}}% 
            \lccode`\~`\/\lowercase{\def~}{\discretionary{\hbox{\char`\/}}{\Wrappedafterbreak}{\hbox{\char`\/}}}% 
            \catcode`\.\active
            \catcode`\,\active 
            \catcode`\;\active
            \catcode`\:\active
            \catcode`\?\active
            \catcode`\!\active
            \catcode`\/\active 
            \lccode`\~`\~ 	
        }
    \makeatother

    \let\OriginalVerbatim=\Verbatim
    \makeatletter
    \renewcommand{\Verbatim}[1][1]{%
        %\parskip\z@skip
        \sbox\Wrappedcontinuationbox {\Wrappedcontinuationsymbol}%
        \sbox\Wrappedvisiblespacebox {\FV@SetupFont\Wrappedvisiblespace}%
        \def\FancyVerbFormatLine ##1{\hsize\linewidth
            \vtop{\raggedright\hyphenpenalty\z@\exhyphenpenalty\z@
                \doublehyphendemerits\z@\finalhyphendemerits\z@
                \strut ##1\strut}%
        }%
        % If the linebreak is at a space, the latter will be displayed as visible
        % space at end of first line, and a continuation symbol starts next line.
        % Stretch/shrink are however usually zero for typewriter font.
        \def\FV@Space {%
            \nobreak\hskip\z@ plus\fontdimen3\font minus\fontdimen4\font
            \discretionary{\copy\Wrappedvisiblespacebox}{\Wrappedafterbreak}
            {\kern\fontdimen2\font}%
        }%
        
        % Allow breaks at special characters using \PYG... macros.
        \Wrappedbreaksatspecials
        % Breaks at punctuation characters . , ; ? ! and / need catcode=\active 	
        \OriginalVerbatim[#1,codes*=\Wrappedbreaksatpunct]%
    }
    \makeatother

    % Exact colors from NB
    \definecolor{incolor}{HTML}{303F9F}
    \definecolor{outcolor}{HTML}{D84315}
    \definecolor{cellborder}{HTML}{CFCFCF}
    \definecolor{cellbackground}{HTML}{F7F7F7}
    
    % prompt
    \makeatletter
    \newcommand{\boxspacing}{\kern\kvtcb@left@rule\kern\kvtcb@boxsep}
    \makeatother
    \newcommand{\prompt}[4]{
        \ttfamily\llap{{\color{#2}[#3]:\hspace{3pt}#4}}\vspace{-\baselineskip}
    }
    

    
    % Prevent overflowing lines due to hard-to-break entities
    \sloppy 
    % Setup hyperref package
    \hypersetup{
      breaklinks=true,  % so long urls are correctly broken across lines
      colorlinks=true,
      urlcolor=urlcolor,
      linkcolor=linkcolor,
      citecolor=citecolor,
      }
    % Slightly bigger margins than the latex defaults
    
    \geometry{verbose,tmargin=1in,bmargin=1in,lmargin=1in,rmargin=1in}
    
    

\begin{document}
    
    \maketitle
    
    

    
    \hypertarget{creating-a-sampled-dataset}{%
\section{Creating a Sampled Dataset}\label{creating-a-sampled-dataset}}

\textbf{Learning Objectives}

\begin{enumerate}
\def\labelenumi{\arabic{enumi}.}
\tightlist
\item
  Setup up the environment
\item
  Sample the natality dataset to create train, eval, test sets
\item
  Preprocess the data in Pandas dataframe
\end{enumerate}

\hypertarget{introduction}{%
\subsection{Introduction}\label{introduction}}

In this notebook, we'll read data from BigQuery into our notebook to
preprocess the data within a Pandas dataframe for a small, repeatable
sample.

We will set up the environment, sample the natality dataset to create
train, eval, test splits, and preprocess the data in a Pandas dataframe.

Each learning objective will correspond to a \textbf{\#TODO} in this
student lab notebook -- try to complete this notebook first and then
review the
\href{https://github.com/GoogleCloudPlatform/training-data-analyst/tree/master/courses/machine_learning/deepdive2/end_to_end_ml/solutions/sample_babyweight.ipynb}{solution
notebook}.

    \hypertarget{set-up-environment-variables-and-load-necessary-libraries}{%
\subsection{Set up environment variables and load necessary
libraries}\label{set-up-environment-variables-and-load-necessary-libraries}}

    \begin{tcolorbox}[breakable, size=fbox, boxrule=1pt, pad at break*=1mm,colback=cellbackground, colframe=cellborder]
\prompt{In}{incolor}{1}{\boxspacing}
\begin{Verbatim}[commandchars=\\\{\}]
\PY{o}{!}sudo chown \PYZhy{}R jupyter:jupyter /home/jupyter/training\PYZhy{}data\PYZhy{}analyst
\end{Verbatim}
\end{tcolorbox}

    \begin{tcolorbox}[breakable, size=fbox, boxrule=1pt, pad at break*=1mm,colback=cellbackground, colframe=cellborder]
\prompt{In}{incolor}{2}{\boxspacing}
\begin{Verbatim}[commandchars=\\\{\}]
\PY{o}{!}pip install \PYZhy{}\PYZhy{}user google\PYZhy{}cloud\PYZhy{}bigquery\PY{o}{=}\PY{o}{=}\PY{l+m}{1}.25.0
\end{Verbatim}
\end{tcolorbox}

    \begin{Verbatim}[commandchars=\\\{\}]
Collecting google-cloud-bigquery==1.25.0
  Downloading google\_cloud\_bigquery-1.25.0-py2.py3-none-any.whl (169 kB)
     |████████████████████████████████| 169 kB 8.3 MB/s eta 0:00:01
Requirement already satisfied: protobuf>=3.6.0 in
/opt/conda/lib/python3.7/site-packages (from google-cloud-bigquery==1.25.0)
(3.13.0)
Requirement already satisfied: google-auth<2.0dev,>=1.9.0 in
/opt/conda/lib/python3.7/site-packages (from google-cloud-bigquery==1.25.0)
(1.23.0)
Collecting google-resumable-media<0.6dev,>=0.5.0
  Downloading google\_resumable\_media-0.5.1-py2.py3-none-any.whl (38 kB)
Requirement already satisfied: google-cloud-core<2.0dev,>=1.1.0 in
/opt/conda/lib/python3.7/site-packages (from google-cloud-bigquery==1.25.0)
(1.3.0)
Requirement already satisfied: six<2.0.0dev,>=1.13.0 in
/opt/conda/lib/python3.7/site-packages (from google-cloud-bigquery==1.25.0)
(1.15.0)
Requirement already satisfied: google-api-core<2.0dev,>=1.15.0 in
/opt/conda/lib/python3.7/site-packages (from google-cloud-bigquery==1.25.0)
(1.22.4)
Requirement already satisfied: setuptools in /opt/conda/lib/python3.7/site-
packages (from protobuf>=3.6.0->google-cloud-bigquery==1.25.0) (50.3.2)
Requirement already satisfied: cachetools<5.0,>=2.0.0 in
/opt/conda/lib/python3.7/site-packages (from google-auth<2.0dev,>=1.9.0->google-
cloud-bigquery==1.25.0) (4.1.1)
Requirement already satisfied: pyasn1-modules>=0.2.1 in
/opt/conda/lib/python3.7/site-packages (from google-auth<2.0dev,>=1.9.0->google-
cloud-bigquery==1.25.0) (0.2.8)
Requirement already satisfied: rsa<5,>=3.1.4; python\_version >= "3.5" in
/opt/conda/lib/python3.7/site-packages (from google-auth<2.0dev,>=1.9.0->google-
cloud-bigquery==1.25.0) (4.6)
Requirement already satisfied: googleapis-common-protos<2.0dev,>=1.6.0 in
/opt/conda/lib/python3.7/site-packages (from google-api-
core<2.0dev,>=1.15.0->google-cloud-bigquery==1.25.0) (1.52.0)
Requirement already satisfied: pytz in /opt/conda/lib/python3.7/site-packages
(from google-api-core<2.0dev,>=1.15.0->google-cloud-bigquery==1.25.0) (2020.4)
Requirement already satisfied: requests<3.0.0dev,>=2.18.0 in
/opt/conda/lib/python3.7/site-packages (from google-api-
core<2.0dev,>=1.15.0->google-cloud-bigquery==1.25.0) (2.24.0)
Requirement already satisfied: pyasn1<0.5.0,>=0.4.6 in
/opt/conda/lib/python3.7/site-packages (from pyasn1-modules>=0.2.1->google-
auth<2.0dev,>=1.9.0->google-cloud-bigquery==1.25.0) (0.4.8)
Requirement already satisfied: chardet<4,>=3.0.2 in
/opt/conda/lib/python3.7/site-packages (from requests<3.0.0dev,>=2.18.0->google-
api-core<2.0dev,>=1.15.0->google-cloud-bigquery==1.25.0) (3.0.4)
Requirement already satisfied: idna<3,>=2.5 in /opt/conda/lib/python3.7/site-
packages (from requests<3.0.0dev,>=2.18.0->google-api-
core<2.0dev,>=1.15.0->google-cloud-bigquery==1.25.0) (2.10)
Requirement already satisfied: certifi>=2017.4.17 in
/opt/conda/lib/python3.7/site-packages (from requests<3.0.0dev,>=2.18.0->google-
api-core<2.0dev,>=1.15.0->google-cloud-bigquery==1.25.0) (2020.11.8)
Requirement already satisfied: urllib3!=1.25.0,!=1.25.1,<1.26,>=1.21.1 in
/opt/conda/lib/python3.7/site-packages (from requests<3.0.0dev,>=2.18.0->google-
api-core<2.0dev,>=1.15.0->google-cloud-bigquery==1.25.0) (1.25.11)
Installing collected packages: google-resumable-media, google-cloud-bigquery
\textcolor{ansi-red}{ERROR: After October 2020 you may experience errors when installing or
updating packages. This is because pip will change the way that it resolves
dependency conflicts.

We recommend you use --use-feature=2020-resolver to test your packages with the
new resolver before it becomes the default.

tfx 0.23.0 requires attrs<20,>=19.3.0, but you'll have attrs 20.3.0 which is
incompatible.
tfx 0.23.0 requires google-resumable-media<0.7.0,>=0.6.0, but you'll have
google-resumable-media 0.5.1 which is incompatible.
tfx 0.23.0 requires kubernetes<12,>=10.0.1, but you'll have kubernetes 12.0.0
which is incompatible.
tfx 0.23.0 requires pyarrow<0.18,>=0.17, but you'll have pyarrow 2.0.0 which is
incompatible.
google-cloud-storage 1.30.0 requires google-resumable-media<2.0dev,>=0.6.0, but
you'll have google-resumable-media 0.5.1 which is incompatible.}
Successfully installed google-cloud-bigquery-1.25.0 google-resumable-media-0.5.1
    \end{Verbatim}

    \textbf{Note}: Restart your kernel to use updated packages.

    Kindly ignore the deprecation warnings and incompatibility errors
related to google-cloud-storage.

    Import necessary libraries.

    \begin{tcolorbox}[breakable, size=fbox, boxrule=1pt, pad at break*=1mm,colback=cellbackground, colframe=cellborder]
\prompt{In}{incolor}{1}{\boxspacing}
\begin{Verbatim}[commandchars=\\\{\}]
\PY{k+kn}{from} \PY{n+nn}{google}\PY{n+nn}{.}\PY{n+nn}{cloud} \PY{k+kn}{import} \PY{n}{bigquery}
\PY{k+kn}{import} \PY{n+nn}{pandas} \PY{k}{as} \PY{n+nn}{pd}
\end{Verbatim}
\end{tcolorbox}

    \textbf{Lab Task \#1:} Set up environment variables so that we can use
them throughout the notebook

    \begin{tcolorbox}[breakable, size=fbox, boxrule=1pt, pad at break*=1mm,colback=cellbackground, colframe=cellborder]
\prompt{In}{incolor}{2}{\boxspacing}
\begin{Verbatim}[commandchars=\\\{\}]
\PY{o}{\PYZpc{}\PYZpc{}}\PY{k}{bash}
\PYZsh{} TODO 1
\PYZsh{} TODO \PYZhy{}\PYZhy{} Your code here.
echo \PYZdq{}Your current GCP Project Name is: \PYZdq{}\PYZdl{}PROJECT
\end{Verbatim}
\end{tcolorbox}

    \begin{Verbatim}[commandchars=\\\{\}]
Your current GCP Project Name is:
    \end{Verbatim}

    \begin{tcolorbox}[breakable, size=fbox, boxrule=1pt, pad at break*=1mm,colback=cellbackground, colframe=cellborder]
\prompt{In}{incolor}{3}{\boxspacing}
\begin{Verbatim}[commandchars=\\\{\}]
\PY{n}{PROJECT} \PY{o}{=} \PY{l+s+s2}{\PYZdq{}}\PY{l+s+s2}{qwiklabs\PYZhy{}gcp\PYZhy{}03\PYZhy{}5d3dc033c852}\PY{l+s+s2}{\PYZdq{}}  \PY{c+c1}{\PYZsh{} Replace with your PROJECT}
\end{Verbatim}
\end{tcolorbox}

    \hypertarget{create-ml-datasets-by-sampling-using-bigquery}{%
\subsection{Create ML datasets by sampling using
BigQuery}\label{create-ml-datasets-by-sampling-using-bigquery}}

We'll begin by sampling the BigQuery data to create smaller datasets.
Let's create a BigQuery client that we'll use throughout the lab.

    \begin{tcolorbox}[breakable, size=fbox, boxrule=1pt, pad at break*=1mm,colback=cellbackground, colframe=cellborder]
\prompt{In}{incolor}{4}{\boxspacing}
\begin{Verbatim}[commandchars=\\\{\}]
\PY{n}{bq} \PY{o}{=} \PY{n}{bigquery}\PY{o}{.}\PY{n}{Client}\PY{p}{(}\PY{n}{project} \PY{o}{=} \PY{n}{PROJECT}\PY{p}{)}
\end{Verbatim}
\end{tcolorbox}

    We need to figure out the right way to divide our hash values to get our
desired splits. To do that we need to define some values to hash within
the module. Feel free to play around with these values to get the
perfect combination.

    \begin{tcolorbox}[breakable, size=fbox, boxrule=1pt, pad at break*=1mm,colback=cellbackground, colframe=cellborder]
\prompt{In}{incolor}{5}{\boxspacing}
\begin{Verbatim}[commandchars=\\\{\}]
\PY{n}{modulo\PYZus{}divisor} \PY{o}{=} \PY{l+m+mi}{100}
\PY{n}{train\PYZus{}percent} \PY{o}{=} \PY{l+m+mf}{80.0}
\PY{n}{eval\PYZus{}percent} \PY{o}{=} \PY{l+m+mf}{10.0}

\PY{n}{train\PYZus{}buckets} \PY{o}{=} \PY{n+nb}{int}\PY{p}{(}\PY{n}{modulo\PYZus{}divisor} \PY{o}{*} \PY{n}{train\PYZus{}percent} \PY{o}{/} \PY{l+m+mf}{100.0}\PY{p}{)}
\PY{n}{eval\PYZus{}buckets} \PY{o}{=} \PY{n+nb}{int}\PY{p}{(}\PY{n}{modulo\PYZus{}divisor} \PY{o}{*} \PY{n}{eval\PYZus{}percent} \PY{o}{/} \PY{l+m+mf}{100.0}\PY{p}{)}
\end{Verbatim}
\end{tcolorbox}

    We can make a series of queries to check if our bucketing values result
in the correct sizes of each of our dataset splits and then adjust
accordingly. Therefore, to make our code more compact and reusable,
let's define a function to return the head of a dataframe produced from
our queries up to a certain number of rows.

    \begin{tcolorbox}[breakable, size=fbox, boxrule=1pt, pad at break*=1mm,colback=cellbackground, colframe=cellborder]
\prompt{In}{incolor}{6}{\boxspacing}
\begin{Verbatim}[commandchars=\\\{\}]
\PY{k}{def} \PY{n+nf}{display\PYZus{}dataframe\PYZus{}head\PYZus{}from\PYZus{}query}\PY{p}{(}\PY{n}{query}\PY{p}{,} \PY{n}{count}\PY{o}{=}\PY{l+m+mi}{10}\PY{p}{)}\PY{p}{:}
    \PY{l+s+sd}{\PYZdq{}\PYZdq{}\PYZdq{}Displays count rows from dataframe head from query.}
\PY{l+s+sd}{    }
\PY{l+s+sd}{    Args:}
\PY{l+s+sd}{        query: str, query to be run on BigQuery, results stored in dataframe.}
\PY{l+s+sd}{        count: int, number of results from head of dataframe to display.}
\PY{l+s+sd}{    Returns:}
\PY{l+s+sd}{        Dataframe head with count number of results.}
\PY{l+s+sd}{    \PYZdq{}\PYZdq{}\PYZdq{}}
    \PY{n}{df} \PY{o}{=} \PY{n}{bq}\PY{o}{.}\PY{n}{query}\PY{p}{(}
        \PY{n}{query} \PY{o}{+} \PY{l+s+s2}{\PYZdq{}}\PY{l+s+s2}{ LIMIT }\PY{l+s+si}{\PYZob{}limit\PYZcb{}}\PY{l+s+s2}{\PYZdq{}}\PY{o}{.}\PY{n}{format}\PY{p}{(}
            \PY{n}{limit}\PY{o}{=}\PY{n}{count}\PY{p}{)}\PY{p}{)}\PY{o}{.}\PY{n}{to\PYZus{}dataframe}\PY{p}{(}\PY{p}{)}

    \PY{k}{return} \PY{n}{df}\PY{o}{.}\PY{n}{head}\PY{p}{(}\PY{n}{count}\PY{p}{)}
\end{Verbatim}
\end{tcolorbox}

    For our first query, we're going to use the original query above to get
our label, features, and columns to combine into our hash which we will
use to perform our repeatable splitting. There are only a limited number
of years, months, days, and states in the dataset. Let's see what the
hash values are. We will need to include all of these extra columns to
hash on to get a fairly uniform spread of the data. Feel free to try
less or more in the hash and see how it changes your results.

    \begin{tcolorbox}[breakable, size=fbox, boxrule=1pt, pad at break*=1mm,colback=cellbackground, colframe=cellborder]
\prompt{In}{incolor}{7}{\boxspacing}
\begin{Verbatim}[commandchars=\\\{\}]
\PY{c+c1}{\PYZsh{} Get label, features, and columns to hash and split into buckets}
\PY{n}{hash\PYZus{}cols\PYZus{}fixed\PYZus{}query} \PY{o}{=} \PY{l+s+s2}{\PYZdq{}\PYZdq{}\PYZdq{}}
\PY{l+s+s2}{SELECT}
\PY{l+s+s2}{    weight\PYZus{}pounds,}
\PY{l+s+s2}{    is\PYZus{}male,}
\PY{l+s+s2}{    mother\PYZus{}age,}
\PY{l+s+s2}{    plurality,}
\PY{l+s+s2}{    gestation\PYZus{}weeks,}
\PY{l+s+s2}{    year,}
\PY{l+s+s2}{    month,}
\PY{l+s+s2}{    CASE}
\PY{l+s+s2}{        WHEN day IS NULL THEN}
\PY{l+s+s2}{            CASE}
\PY{l+s+s2}{                WHEN wday IS NULL THEN 0}
\PY{l+s+s2}{                ELSE wday}
\PY{l+s+s2}{            END}
\PY{l+s+s2}{        ELSE day}
\PY{l+s+s2}{    END AS date,}
\PY{l+s+s2}{    IFNULL(state, }\PY{l+s+s2}{\PYZdq{}}\PY{l+s+s2}{Unknown}\PY{l+s+s2}{\PYZdq{}}\PY{l+s+s2}{) AS state,}
\PY{l+s+s2}{    IFNULL(mother\PYZus{}birth\PYZus{}state, }\PY{l+s+s2}{\PYZdq{}}\PY{l+s+s2}{Unknown}\PY{l+s+s2}{\PYZdq{}}\PY{l+s+s2}{) AS mother\PYZus{}birth\PYZus{}state}
\PY{l+s+s2}{FROM}
\PY{l+s+s2}{    publicdata.samples.natality}
\PY{l+s+s2}{WHERE}
\PY{l+s+s2}{    year \PYZgt{} 2000}
\PY{l+s+s2}{    AND weight\PYZus{}pounds \PYZgt{} 0}
\PY{l+s+s2}{    AND mother\PYZus{}age \PYZgt{} 0}
\PY{l+s+s2}{    AND plurality \PYZgt{} 0}
\PY{l+s+s2}{    AND gestation\PYZus{}weeks \PYZgt{} 0}
\PY{l+s+s2}{\PYZdq{}\PYZdq{}\PYZdq{}}

\PY{n}{display\PYZus{}dataframe\PYZus{}head\PYZus{}from\PYZus{}query}\PY{p}{(}\PY{n}{hash\PYZus{}cols\PYZus{}fixed\PYZus{}query}\PY{p}{)}
\end{Verbatim}
\end{tcolorbox}

            \begin{tcolorbox}[breakable, size=fbox, boxrule=.5pt, pad at break*=1mm, opacityfill=0]
\prompt{Out}{outcolor}{7}{\boxspacing}
\begin{Verbatim}[commandchars=\\\{\}]
   weight\_pounds  is\_male  mother\_age  plurality  gestation\_weeks  year  \textbackslash{}
0       7.063611     True          32          1               37  2001
1       4.687028     True          30          3               33  2001
2       7.561856     True          20          1               39  2001
3       7.561856     True          31          1               37  2001
4       7.312733     True          32          1               40  2001
5       7.627994    False          30          1               40  2001
6       7.251004     True          33          1               37  2001
7       7.500126    False          23          1               39  2001
8       7.125340    False          33          1               39  2001
9       7.749249     True          31          1               39  2001

   month  date state mother\_birth\_state
0     12     3    CO                 CA
1      6     5    IN                 IN
2      4     5    MN                 MN
3     10     5    MS                 MS
4     11     3    MO                 MO
5     10     5    NY                 PA
6     11     5    WA                 WA
7      9     2    OK                 LA
8      1     4    TX                 MS
9      1     1    TX            Foreign
\end{Verbatim}
\end{tcolorbox}
        
    Using \texttt{COALESCE} would provide the same result as the nested
\texttt{CASE\ WHEN}. This is preferable when all we want is the first
non-null instance. To be precise the \texttt{CASE\ WHEN} would become
\texttt{COALESCE(wday,\ day,\ 0)\ AS\ date}. You can read more about it
\href{https://cloud.google.com/bigquery/docs/reference/standard-sql/conditional_expressions}{here}.

    Next query will combine our hash columns and will leave us just with our
label, features, and our hash values.

    \begin{tcolorbox}[breakable, size=fbox, boxrule=1pt, pad at break*=1mm,colback=cellbackground, colframe=cellborder]
\prompt{In}{incolor}{8}{\boxspacing}
\begin{Verbatim}[commandchars=\\\{\}]
\PY{n}{data\PYZus{}query} \PY{o}{=} \PY{l+s+s2}{\PYZdq{}\PYZdq{}\PYZdq{}}
\PY{l+s+s2}{SELECT}
\PY{l+s+s2}{    weight\PYZus{}pounds,}
\PY{l+s+s2}{    is\PYZus{}male,}
\PY{l+s+s2}{    mother\PYZus{}age,}
\PY{l+s+s2}{    plurality,}
\PY{l+s+s2}{    gestation\PYZus{}weeks,}
\PY{l+s+s2}{    FARM\PYZus{}FINGERPRINT(}
\PY{l+s+s2}{        CONCAT(}
\PY{l+s+s2}{            CAST(year AS STRING),}
\PY{l+s+s2}{            CAST(month AS STRING),}
\PY{l+s+s2}{            CAST(date AS STRING),}
\PY{l+s+s2}{            CAST(state AS STRING),}
\PY{l+s+s2}{            CAST(mother\PYZus{}birth\PYZus{}state AS STRING)}
\PY{l+s+s2}{        )}
\PY{l+s+s2}{    ) AS hash\PYZus{}values}
\PY{l+s+s2}{FROM}
\PY{l+s+s2}{    (}\PY{l+s+si}{\PYZob{}CTE\PYZus{}hash\PYZus{}cols\PYZus{}fixed\PYZcb{}}\PY{l+s+s2}{)}
\PY{l+s+s2}{\PYZdq{}\PYZdq{}\PYZdq{}}\PY{o}{.}\PY{n}{format}\PY{p}{(}\PY{n}{CTE\PYZus{}hash\PYZus{}cols\PYZus{}fixed}\PY{o}{=}\PY{n}{hash\PYZus{}cols\PYZus{}fixed\PYZus{}query}\PY{p}{)}

\PY{n}{display\PYZus{}dataframe\PYZus{}head\PYZus{}from\PYZus{}query}\PY{p}{(}\PY{n}{data\PYZus{}query}\PY{p}{)}
\end{Verbatim}
\end{tcolorbox}

            \begin{tcolorbox}[breakable, size=fbox, boxrule=.5pt, pad at break*=1mm, opacityfill=0]
\prompt{Out}{outcolor}{8}{\boxspacing}
\begin{Verbatim}[commandchars=\\\{\}]
   weight\_pounds  is\_male  mother\_age  plurality  gestation\_weeks  \textbackslash{}
0       7.063611     True          32          1               37
1       4.687028     True          30          3               33
2       7.561856     True          20          1               39
3       7.561856     True          31          1               37
4       7.312733     True          32          1               40
5       7.627994    False          30          1               40
6       7.251004     True          33          1               37
7       7.500126    False          23          1               39
8       7.125340    False          33          1               39
9       7.749249     True          31          1               39

           hash\_values
0  4762325092919148672
1  2341060194216507348
2 -8842767231851202242
3  7957807816914159435
4 -5961624242430066305
5  5493295634082918412
6 -2988893757655690534
7 -6735199252008114417
8 -3514093303120687641
9  2175328516857391398
\end{Verbatim}
\end{tcolorbox}
        
    The next query is going to find the counts of each of the unique 657484
\texttt{hash\_values}. This will be our first step at making actual hash
buckets for our split via the \texttt{GROUP\ BY}.

    \begin{tcolorbox}[breakable, size=fbox, boxrule=1pt, pad at break*=1mm,colback=cellbackground, colframe=cellborder]
\prompt{In}{incolor}{9}{\boxspacing}
\begin{Verbatim}[commandchars=\\\{\}]
\PY{c+c1}{\PYZsh{} Get the counts of each of the unique hash of our splitting column}
\PY{n}{first\PYZus{}bucketing\PYZus{}query} \PY{o}{=} \PY{l+s+s2}{\PYZdq{}\PYZdq{}\PYZdq{}}
\PY{l+s+s2}{SELECT}
\PY{l+s+s2}{    hash\PYZus{}values,}
\PY{l+s+s2}{    COUNT(*) AS num\PYZus{}records}
\PY{l+s+s2}{FROM}
\PY{l+s+s2}{    (}\PY{l+s+si}{\PYZob{}CTE\PYZus{}data\PYZcb{}}\PY{l+s+s2}{)}
\PY{l+s+s2}{GROUP BY}
\PY{l+s+s2}{    hash\PYZus{}values}
\PY{l+s+s2}{\PYZdq{}\PYZdq{}\PYZdq{}}\PY{o}{.}\PY{n}{format}\PY{p}{(}\PY{n}{CTE\PYZus{}data}\PY{o}{=}\PY{n}{data\PYZus{}query}\PY{p}{)}

\PY{n}{display\PYZus{}dataframe\PYZus{}head\PYZus{}from\PYZus{}query}\PY{p}{(}\PY{n}{first\PYZus{}bucketing\PYZus{}query}\PY{p}{)}
\end{Verbatim}
\end{tcolorbox}

            \begin{tcolorbox}[breakable, size=fbox, boxrule=.5pt, pad at break*=1mm, opacityfill=0]
\prompt{Out}{outcolor}{9}{\boxspacing}
\begin{Verbatim}[commandchars=\\\{\}]
           hash\_values  num\_records
0 -1700820252994836306          741
1  7948408306271784936          883
2 -1740303207227716653          794
3  5160546322234461939         2887
4 -6766011436514751671          178
5 -3974717950920322290           78
6 -8405995084719745013            2
7  8641800065663952690          875
8 -2290255568977475467          212
9  -646460941575642964          463
\end{Verbatim}
\end{tcolorbox}
        
    The query below performs a second layer of bucketing where now for each
of these bucket indices we count the number of records.

    \begin{tcolorbox}[breakable, size=fbox, boxrule=1pt, pad at break*=1mm,colback=cellbackground, colframe=cellborder]
\prompt{In}{incolor}{10}{\boxspacing}
\begin{Verbatim}[commandchars=\\\{\}]
\PY{c+c1}{\PYZsh{} Get the number of records in each of the hash buckets}
\PY{n}{second\PYZus{}bucketing\PYZus{}query} \PY{o}{=} \PY{l+s+s2}{\PYZdq{}\PYZdq{}\PYZdq{}}
\PY{l+s+s2}{SELECT}
\PY{l+s+s2}{    ABS(MOD(hash\PYZus{}values, }\PY{l+s+si}{\PYZob{}modulo\PYZus{}divisor\PYZcb{}}\PY{l+s+s2}{)) AS bucket\PYZus{}index,}
\PY{l+s+s2}{    SUM(num\PYZus{}records) AS num\PYZus{}records}
\PY{l+s+s2}{FROM}
\PY{l+s+s2}{    (}\PY{l+s+si}{\PYZob{}CTE\PYZus{}first\PYZus{}bucketing\PYZcb{}}\PY{l+s+s2}{)}
\PY{l+s+s2}{GROUP BY}
\PY{l+s+s2}{    ABS(MOD(hash\PYZus{}values, }\PY{l+s+si}{\PYZob{}modulo\PYZus{}divisor\PYZcb{}}\PY{l+s+s2}{))}
\PY{l+s+s2}{\PYZdq{}\PYZdq{}\PYZdq{}}\PY{o}{.}\PY{n}{format}\PY{p}{(}
    \PY{n}{CTE\PYZus{}first\PYZus{}bucketing}\PY{o}{=}\PY{n}{first\PYZus{}bucketing\PYZus{}query}\PY{p}{,} \PY{n}{modulo\PYZus{}divisor}\PY{o}{=}\PY{n}{modulo\PYZus{}divisor}\PY{p}{)}

\PY{n}{display\PYZus{}dataframe\PYZus{}head\PYZus{}from\PYZus{}query}\PY{p}{(}\PY{n}{second\PYZus{}bucketing\PYZus{}query}\PY{p}{)}
\end{Verbatim}
\end{tcolorbox}

            \begin{tcolorbox}[breakable, size=fbox, boxrule=.5pt, pad at break*=1mm, opacityfill=0]
\prompt{Out}{outcolor}{10}{\boxspacing}
\begin{Verbatim}[commandchars=\\\{\}]
   bucket\_index  num\_records
0            62       426834
1            46       281627
2            76       354090
3            87       523881
4             0       277395
5            63       355283
6            58       209618
7            66       402627
8            34       379000
9            56       226752
\end{Verbatim}
\end{tcolorbox}
        
    The number of records is hard for us to easily understand the split, so
we will normalize the count into percentage of the data in each of the
hash buckets in the next query.

    \begin{tcolorbox}[breakable, size=fbox, boxrule=1pt, pad at break*=1mm,colback=cellbackground, colframe=cellborder]
\prompt{In}{incolor}{11}{\boxspacing}
\begin{Verbatim}[commandchars=\\\{\}]
\PY{c+c1}{\PYZsh{} Calculate the overall percentages}
\PY{n}{percentages\PYZus{}query} \PY{o}{=} \PY{l+s+s2}{\PYZdq{}\PYZdq{}\PYZdq{}}
\PY{l+s+s2}{SELECT}
\PY{l+s+s2}{    bucket\PYZus{}index,}
\PY{l+s+s2}{    num\PYZus{}records,}
\PY{l+s+s2}{    CAST(num\PYZus{}records AS FLOAT64) / (}
\PY{l+s+s2}{    SELECT}
\PY{l+s+s2}{        SUM(num\PYZus{}records)}
\PY{l+s+s2}{    FROM}
\PY{l+s+s2}{        (}\PY{l+s+si}{\PYZob{}CTE\PYZus{}second\PYZus{}bucketing\PYZcb{}}\PY{l+s+s2}{)) AS percent\PYZus{}records}
\PY{l+s+s2}{FROM}
\PY{l+s+s2}{    (}\PY{l+s+si}{\PYZob{}CTE\PYZus{}second\PYZus{}bucketing\PYZcb{}}\PY{l+s+s2}{)}
\PY{l+s+s2}{\PYZdq{}\PYZdq{}\PYZdq{}}\PY{o}{.}\PY{n}{format}\PY{p}{(}\PY{n}{CTE\PYZus{}second\PYZus{}bucketing}\PY{o}{=}\PY{n}{second\PYZus{}bucketing\PYZus{}query}\PY{p}{)}

\PY{n}{display\PYZus{}dataframe\PYZus{}head\PYZus{}from\PYZus{}query}\PY{p}{(}\PY{n}{percentages\PYZus{}query}\PY{p}{)}
\end{Verbatim}
\end{tcolorbox}

            \begin{tcolorbox}[breakable, size=fbox, boxrule=.5pt, pad at break*=1mm, opacityfill=0]
\prompt{Out}{outcolor}{11}{\boxspacing}
\begin{Verbatim}[commandchars=\\\{\}]
   bucket\_index  num\_records  percent\_records
0            70       285539         0.008650
1            91       333267         0.010096
2            78       326758         0.009898
3             0       277395         0.008403
4             4       398118         0.012060
5             9       236637         0.007168
6            33       410226         0.012427
7             6       548778         0.016624
8            84       341155         0.010334
9            38       338150         0.010243
\end{Verbatim}
\end{tcolorbox}
        
    We'll now select the range of buckets to be used in training.

    \begin{tcolorbox}[breakable, size=fbox, boxrule=1pt, pad at break*=1mm,colback=cellbackground, colframe=cellborder]
\prompt{In}{incolor}{12}{\boxspacing}
\begin{Verbatim}[commandchars=\\\{\}]
\PY{c+c1}{\PYZsh{} Choose hash buckets for training and pull in their statistics}
\PY{n}{train\PYZus{}query} \PY{o}{=} \PY{l+s+s2}{\PYZdq{}\PYZdq{}\PYZdq{}}
\PY{l+s+s2}{SELECT}
\PY{l+s+s2}{    *,}
\PY{l+s+s2}{    }\PY{l+s+s2}{\PYZdq{}}\PY{l+s+s2}{train}\PY{l+s+s2}{\PYZdq{}}\PY{l+s+s2}{ AS dataset\PYZus{}name}
\PY{l+s+s2}{FROM}
\PY{l+s+s2}{    (}\PY{l+s+si}{\PYZob{}CTE\PYZus{}percentages\PYZcb{}}\PY{l+s+s2}{)}
\PY{l+s+s2}{WHERE}
\PY{l+s+s2}{    bucket\PYZus{}index \PYZgt{}= 0}
\PY{l+s+s2}{    AND bucket\PYZus{}index \PYZlt{} }\PY{l+s+si}{\PYZob{}train\PYZus{}buckets\PYZcb{}}
\PY{l+s+s2}{\PYZdq{}\PYZdq{}\PYZdq{}}\PY{o}{.}\PY{n}{format}\PY{p}{(}
    \PY{n}{CTE\PYZus{}percentages}\PY{o}{=}\PY{n}{percentages\PYZus{}query}\PY{p}{,}
    \PY{n}{train\PYZus{}buckets}\PY{o}{=}\PY{n}{train\PYZus{}buckets}\PY{p}{)}

\PY{n}{display\PYZus{}dataframe\PYZus{}head\PYZus{}from\PYZus{}query}\PY{p}{(}\PY{n}{train\PYZus{}query}\PY{p}{)}
\end{Verbatim}
\end{tcolorbox}

            \begin{tcolorbox}[breakable, size=fbox, boxrule=.5pt, pad at break*=1mm, opacityfill=0]
\prompt{Out}{outcolor}{12}{\boxspacing}
\begin{Verbatim}[commandchars=\\\{\}]
   bucket\_index  num\_records  percent\_records dataset\_name
0             9       236637         0.007168        train
1            30       333513         0.010103        train
2            20       432535         0.013103        train
3            15       263367         0.007978        train
4            39       224255         0.006793        train
5             2       492473         0.014918        train
6            45       265930         0.008056        train
7            33       410226         0.012427        train
8            53       230298         0.006976        train
9            73       411771         0.012474        train
\end{Verbatim}
\end{tcolorbox}
        
    We'll do the same by selecting the range of buckets to be used
evaluation.

    \begin{tcolorbox}[breakable, size=fbox, boxrule=1pt, pad at break*=1mm,colback=cellbackground, colframe=cellborder]
\prompt{In}{incolor}{13}{\boxspacing}
\begin{Verbatim}[commandchars=\\\{\}]
\PY{c+c1}{\PYZsh{} Choose hash buckets for validation and pull in their statistics}
\PY{n}{eval\PYZus{}query} \PY{o}{=} \PY{l+s+s2}{\PYZdq{}\PYZdq{}\PYZdq{}}
\PY{l+s+s2}{SELECT}
\PY{l+s+s2}{    *,}
\PY{l+s+s2}{    }\PY{l+s+s2}{\PYZdq{}}\PY{l+s+s2}{eval}\PY{l+s+s2}{\PYZdq{}}\PY{l+s+s2}{ AS dataset\PYZus{}name}
\PY{l+s+s2}{FROM}
\PY{l+s+s2}{    (}\PY{l+s+si}{\PYZob{}CTE\PYZus{}percentages\PYZcb{}}\PY{l+s+s2}{)}
\PY{l+s+s2}{WHERE}
\PY{l+s+s2}{    bucket\PYZus{}index \PYZgt{}= }\PY{l+s+si}{\PYZob{}train\PYZus{}buckets\PYZcb{}}
\PY{l+s+s2}{    AND bucket\PYZus{}index \PYZlt{} }\PY{l+s+si}{\PYZob{}cum\PYZus{}eval\PYZus{}buckets\PYZcb{}}
\PY{l+s+s2}{\PYZdq{}\PYZdq{}\PYZdq{}}\PY{o}{.}\PY{n}{format}\PY{p}{(}
    \PY{n}{CTE\PYZus{}percentages}\PY{o}{=}\PY{n}{percentages\PYZus{}query}\PY{p}{,}
    \PY{n}{train\PYZus{}buckets}\PY{o}{=}\PY{n}{train\PYZus{}buckets}\PY{p}{,}
    \PY{n}{cum\PYZus{}eval\PYZus{}buckets}\PY{o}{=}\PY{n}{train\PYZus{}buckets} \PY{o}{+} \PY{n}{eval\PYZus{}buckets}\PY{p}{)}

\PY{n}{display\PYZus{}dataframe\PYZus{}head\PYZus{}from\PYZus{}query}\PY{p}{(}\PY{n}{eval\PYZus{}query}\PY{p}{)}
\end{Verbatim}
\end{tcolorbox}

            \begin{tcolorbox}[breakable, size=fbox, boxrule=.5pt, pad at break*=1mm, opacityfill=0]
\prompt{Out}{outcolor}{13}{\boxspacing}
\begin{Verbatim}[commandchars=\\\{\}]
   bucket\_index  num\_records  percent\_records dataset\_name
0            88       423809         0.012838         eval
1            85       368045         0.011149         eval
2            87       523881         0.015870         eval
3            89       256482         0.007770         eval
4            82       468179         0.014182         eval
5            84       341155         0.010334         eval
6            80       312489         0.009466         eval
7            83       411258         0.012458         eval
8            81       233538         0.007074         eval
9            86       274489         0.008315         eval
\end{Verbatim}
\end{tcolorbox}
        
    Lastly, we'll select the hash buckets to be used for the test split.

    \begin{tcolorbox}[breakable, size=fbox, boxrule=1pt, pad at break*=1mm,colback=cellbackground, colframe=cellborder]
\prompt{In}{incolor}{15}{\boxspacing}
\begin{Verbatim}[commandchars=\\\{\}]
\PY{c+c1}{\PYZsh{} Choose hash buckets for testing and pull in their statistics}
\PY{n}{test\PYZus{}query} \PY{o}{=} \PY{l+s+s2}{\PYZdq{}\PYZdq{}\PYZdq{}}
\PY{l+s+s2}{SELECT}
\PY{l+s+s2}{    *,}
\PY{l+s+s2}{    }\PY{l+s+s2}{\PYZdq{}}\PY{l+s+s2}{test}\PY{l+s+s2}{\PYZdq{}}\PY{l+s+s2}{ AS dataset\PYZus{}name}
\PY{l+s+s2}{FROM}
\PY{l+s+s2}{    (}\PY{l+s+si}{\PYZob{}CTE\PYZus{}percentages\PYZcb{}}\PY{l+s+s2}{)}
\PY{l+s+s2}{WHERE}
\PY{l+s+s2}{    bucket\PYZus{}index \PYZgt{}= }\PY{l+s+si}{\PYZob{}cum\PYZus{}eval\PYZus{}buckets\PYZcb{}}
\PY{l+s+s2}{    AND bucket\PYZus{}index \PYZlt{} }\PY{l+s+si}{\PYZob{}modulo\PYZus{}divisor\PYZcb{}}
\PY{l+s+s2}{\PYZdq{}\PYZdq{}\PYZdq{}}\PY{o}{.}\PY{n}{format}\PY{p}{(}
    \PY{n}{CTE\PYZus{}percentages}\PY{o}{=}\PY{n}{percentages\PYZus{}query}\PY{p}{,}
    \PY{n}{cum\PYZus{}eval\PYZus{}buckets}\PY{o}{=}\PY{n}{train\PYZus{}buckets} \PY{o}{+} \PY{n}{eval\PYZus{}buckets}\PY{p}{,}
    \PY{n}{modulo\PYZus{}divisor}\PY{o}{=}\PY{n}{modulo\PYZus{}divisor}\PY{p}{)}

\PY{n}{display\PYZus{}dataframe\PYZus{}head\PYZus{}from\PYZus{}query}\PY{p}{(}\PY{n}{test\PYZus{}query}\PY{p}{)}
\end{Verbatim}
\end{tcolorbox}

            \begin{tcolorbox}[breakable, size=fbox, boxrule=.5pt, pad at break*=1mm, opacityfill=0]
\prompt{Out}{outcolor}{15}{\boxspacing}
\begin{Verbatim}[commandchars=\\\{\}]
   bucket\_index  num\_records  percent\_records dataset\_name
0            96       529357         0.016036         test
1            93       215710         0.006534         test
2            98       374697         0.011351         test
3            97       480790         0.014564         test
4            91       333267         0.010096         test
5            94       431001         0.013056         test
6            90       286465         0.008678         test
7            95       313544         0.009498         test
8            92       336735         0.010201         test
9            99       223334         0.006765         test
\end{Verbatim}
\end{tcolorbox}
        
    In the below query, we'll \texttt{UNION\ ALL} all of the datasets
together so that all three sets of hash buckets will be within one
table. We added \texttt{dataset\_id} so that we can sort on it in the
query after.

    \begin{tcolorbox}[breakable, size=fbox, boxrule=1pt, pad at break*=1mm,colback=cellbackground, colframe=cellborder]
\prompt{In}{incolor}{20}{\boxspacing}
\begin{Verbatim}[commandchars=\\\{\}]
\PY{c+c1}{\PYZsh{} Union the training, validation, and testing dataset statistics}
\PY{n}{union\PYZus{}query} \PY{o}{=} \PY{l+s+s2}{\PYZdq{}\PYZdq{}\PYZdq{}}
\PY{l+s+s2}{SELECT}
\PY{l+s+s2}{    0 AS dataset\PYZus{}id,}
\PY{l+s+s2}{    *}
\PY{l+s+s2}{FROM}
\PY{l+s+s2}{    (}\PY{l+s+si}{\PYZob{}CTE\PYZus{}train\PYZcb{}}\PY{l+s+s2}{)}
\PY{l+s+s2}{UNION ALL}
\PY{l+s+s2}{SELECT}
\PY{l+s+s2}{    1 AS dataset\PYZus{}id,}
\PY{l+s+s2}{    *}
\PY{l+s+s2}{FROM}
\PY{l+s+s2}{    (}\PY{l+s+si}{\PYZob{}CTE\PYZus{}eval\PYZcb{}}\PY{l+s+s2}{)}
\PY{l+s+s2}{UNION ALL}
\PY{l+s+s2}{SELECT}
\PY{l+s+s2}{    2 AS dataset\PYZus{}id,}
\PY{l+s+s2}{    *}
\PY{l+s+s2}{FROM}
\PY{l+s+s2}{    (}\PY{l+s+si}{\PYZob{}CTE\PYZus{}test\PYZcb{}}\PY{l+s+s2}{)}
\PY{l+s+s2}{\PYZdq{}\PYZdq{}\PYZdq{}}\PY{o}{.}\PY{n}{format}\PY{p}{(}\PY{n}{CTE\PYZus{}train}\PY{o}{=}\PY{n}{train\PYZus{}query}\PY{p}{,} \PY{n}{CTE\PYZus{}eval}\PY{o}{=}\PY{n}{eval\PYZus{}query}\PY{p}{,} \PY{n}{CTE\PYZus{}test}\PY{o}{=}\PY{n}{test\PYZus{}query}\PY{p}{)}

\PY{n}{display\PYZus{}dataframe\PYZus{}head\PYZus{}from\PYZus{}query}\PY{p}{(}\PY{n}{union\PYZus{}query}\PY{p}{)}
\end{Verbatim}
\end{tcolorbox}

            \begin{tcolorbox}[breakable, size=fbox, boxrule=.5pt, pad at break*=1mm, opacityfill=0]
\prompt{Out}{outcolor}{20}{\boxspacing}
\begin{Verbatim}[commandchars=\\\{\}]
   dataset\_id  bucket\_index  num\_records  percent\_records dataset\_name
0           0            36       246041         0.007453        train
1           0             3       196889         0.005964        train
2           0            67       372457         0.011283        train
3           0             5       449280         0.013610        train
4           0            35       250505         0.007588        train
5           0            34       379000         0.011481        train
6           0            15       263367         0.007978        train
7           0            71       260774         0.007900        train
8           0             0       277395         0.008403        train
9           0            22       257140         0.007789        train
\end{Verbatim}
\end{tcolorbox}
        
    Lastly, we'll show the final split between train, eval, and test sets.
We can see both the number of records and percent of the total data. It
is really close to that we were hoping to get.

    \begin{tcolorbox}[breakable, size=fbox, boxrule=1pt, pad at break*=1mm,colback=cellbackground, colframe=cellborder]
\prompt{In}{incolor}{21}{\boxspacing}
\begin{Verbatim}[commandchars=\\\{\}]
\PY{c+c1}{\PYZsh{} Show final splitting and associated statistics}
\PY{n}{split\PYZus{}query} \PY{o}{=} \PY{l+s+s2}{\PYZdq{}\PYZdq{}\PYZdq{}}
\PY{l+s+s2}{SELECT}
\PY{l+s+s2}{    dataset\PYZus{}id,}
\PY{l+s+s2}{    dataset\PYZus{}name,}
\PY{l+s+s2}{    SUM(num\PYZus{}records) AS num\PYZus{}records,}
\PY{l+s+s2}{    SUM(percent\PYZus{}records) AS percent\PYZus{}records}
\PY{l+s+s2}{FROM}
\PY{l+s+s2}{    (}\PY{l+s+si}{\PYZob{}CTE\PYZus{}union\PYZcb{}}\PY{l+s+s2}{)}
\PY{l+s+s2}{GROUP BY}
\PY{l+s+s2}{    dataset\PYZus{}id,}
\PY{l+s+s2}{    dataset\PYZus{}name}
\PY{l+s+s2}{ORDER BY}
\PY{l+s+s2}{    dataset\PYZus{}id}
\PY{l+s+s2}{\PYZdq{}\PYZdq{}\PYZdq{}}\PY{o}{.}\PY{n}{format}\PY{p}{(}\PY{n}{CTE\PYZus{}union}\PY{o}{=}\PY{n}{union\PYZus{}query}\PY{p}{)}

\PY{n}{display\PYZus{}dataframe\PYZus{}head\PYZus{}from\PYZus{}query}\PY{p}{(}\PY{n}{split\PYZus{}query}\PY{p}{)}
\end{Verbatim}
\end{tcolorbox}

            \begin{tcolorbox}[breakable, size=fbox, boxrule=.5pt, pad at break*=1mm, opacityfill=0]
\prompt{Out}{outcolor}{21}{\boxspacing}
\begin{Verbatim}[commandchars=\\\{\}]
   dataset\_id dataset\_name  num\_records  percent\_records
0           0        train     25873134         0.783765
1           1         eval      3613325         0.109457
2           2         test      3524900         0.106778
\end{Verbatim}
\end{tcolorbox}
        
    Now that we know that our splitting values produce a good global
splitting on our data, here's a way to get a well-distributed portion of
the data in such a way that the train, eval, test sets do not overlap
and takes a subsample of our global splits.

    \begin{tcolorbox}[breakable, size=fbox, boxrule=1pt, pad at break*=1mm,colback=cellbackground, colframe=cellborder]
\prompt{In}{incolor}{25}{\boxspacing}
\begin{Verbatim}[commandchars=\\\{\}]
\PY{k}{def} \PY{n+nf}{dataframe\PYZus{}from\PYZus{}query}\PY{p}{(}\PY{n}{query}\PY{p}{,} \PY{n}{count}\PY{o}{=}\PY{l+m+mi}{10}\PY{p}{)}\PY{p}{:}
    \PY{l+s+sd}{\PYZdq{}\PYZdq{}\PYZdq{}Displays count rows from dataframe head from query.}
\PY{l+s+sd}{    }
\PY{l+s+sd}{    Args:}
\PY{l+s+sd}{        query: str, query to be run on BigQuery, results stored in dataframe.}
\PY{l+s+sd}{        count: int, number of results from head of dataframe to display.}
\PY{l+s+sd}{    Returns:}
\PY{l+s+sd}{        Dataframe head with count number of results.}
\PY{l+s+sd}{    \PYZdq{}\PYZdq{}\PYZdq{}}
    \PY{n}{df} \PY{o}{=} \PY{n}{bq}\PY{o}{.}\PY{n}{query}\PY{p}{(}
        \PY{n}{query} \PY{o}{+} \PY{l+s+s2}{\PYZdq{}}\PY{l+s+s2}{ LIMIT }\PY{l+s+si}{\PYZob{}limit\PYZcb{}}\PY{l+s+s2}{\PYZdq{}}\PY{o}{.}\PY{n}{format}\PY{p}{(}
            \PY{n}{limit}\PY{o}{=}\PY{n}{count}\PY{p}{)}\PY{p}{)}\PY{o}{.}\PY{n}{to\PYZus{}dataframe}\PY{p}{(}\PY{p}{)}

    \PY{k}{return} \PY{n}{df}
\end{Verbatim}
\end{tcolorbox}

    \begin{tcolorbox}[breakable, size=fbox, boxrule=1pt, pad at break*=1mm,colback=cellbackground, colframe=cellborder]
\prompt{In}{incolor}{41}{\boxspacing}
\begin{Verbatim}[commandchars=\\\{\}]
\PY{c+c1}{\PYZsh{} Get the number of records in each of the hash buckets}
\PY{n}{get\PYZus{}train\PYZus{}df} \PY{o}{=} \PY{l+s+s2}{\PYZdq{}\PYZdq{}\PYZdq{}}
\PY{l+s+s2}{SELECT}
\PY{l+s+s2}{	weight\PYZus{}pounds,}
\PY{l+s+s2}{    is\PYZus{}male,}
\PY{l+s+s2}{    mother\PYZus{}age,}
\PY{l+s+s2}{    plurality,}
\PY{l+s+s2}{    gestation\PYZus{}weeks,    }
\PY{l+s+s2}{FROM}
\PY{l+s+s2}{    (}\PY{l+s+si}{\PYZob{}CTE\PYZus{}first\PYZus{}bucketing\PYZcb{}}\PY{l+s+s2}{)}
\PY{l+s+s2}{WHERE}
\PY{l+s+s2}{     ABS(MOD(hash\PYZus{}values, }\PY{l+s+si}{\PYZob{}modulo\PYZus{}divisor\PYZcb{}}\PY{l+s+s2}{)) \PYZlt{} }\PY{l+s+si}{\PYZob{}train\PYZus{}buckets\PYZcb{}}
\PY{l+s+s2}{\PYZdq{}\PYZdq{}\PYZdq{}}\PY{o}{.}\PY{n}{format}\PY{p}{(}
    \PY{n}{CTE\PYZus{}first\PYZus{}bucketing}\PY{o}{=}\PY{n}{first\PYZus{}bucketing\PYZus{}query}\PY{p}{,} \PY{n}{modulo\PYZus{}divisor}\PY{o}{=}\PY{n}{modulo\PYZus{}divisor}\PY{p}{,}\PY{n}{train\PYZus{}buckets}\PY{o}{=}\PY{n}{train\PYZus{}buckets}\PY{p}{)}

\PY{c+c1}{\PYZsh{} Get the number of records in each of the hash buckets}
\PY{n}{get\PYZus{}eval\PYZus{}df} \PY{o}{=} \PY{l+s+s2}{\PYZdq{}\PYZdq{}\PYZdq{}}
\PY{l+s+s2}{SELECT}
\PY{l+s+s2}{	weight\PYZus{}pounds,}
\PY{l+s+s2}{    is\PYZus{}male,}
\PY{l+s+s2}{    mother\PYZus{}age,}
\PY{l+s+s2}{    plurality,}
\PY{l+s+s2}{    gestation\PYZus{}weeks,    }
\PY{l+s+s2}{FROM}
\PY{l+s+s2}{    (}\PY{l+s+si}{\PYZob{}CTE\PYZus{}first\PYZus{}bucketing\PYZcb{}}\PY{l+s+s2}{)}
\PY{l+s+s2}{WHERE}
\PY{l+s+s2}{     ABS(MOD(hash\PYZus{}values, }\PY{l+s+si}{\PYZob{}modulo\PYZus{}divisor\PYZcb{}}\PY{l+s+s2}{)) \PYZgt{}= }\PY{l+s+si}{\PYZob{}train\PYZus{}buckets\PYZcb{}}
\PY{l+s+s2}{     AND ABS(MOD(hash\PYZus{}values, }\PY{l+s+si}{\PYZob{}modulo\PYZus{}divisor\PYZcb{}}\PY{l+s+s2}{)) \PYZlt{} }\PY{l+s+si}{\PYZob{}cum\PYZus{}eval\PYZus{}buckets\PYZcb{}}
\PY{l+s+s2}{\PYZdq{}\PYZdq{}\PYZdq{}}\PY{o}{.}\PY{n}{format}\PY{p}{(}
    \PY{n}{CTE\PYZus{}first\PYZus{}bucketing}\PY{o}{=}\PY{n}{first\PYZus{}bucketing\PYZus{}query}\PY{p}{,} \PY{n}{modulo\PYZus{}divisor}\PY{o}{=}\PY{n}{modulo\PYZus{}divisor}\PY{p}{,} \PY{n}{train\PYZus{}buckets}\PY{o}{=}\PY{n}{train\PYZus{}buckets}\PY{p}{,}
    \PY{n}{cum\PYZus{}eval\PYZus{}buckets}\PY{o}{=}\PY{n}{train\PYZus{}buckets} \PY{o}{+} \PY{n}{eval\PYZus{}buckets}\PY{p}{)}

\PY{c+c1}{\PYZsh{} Get the number of records in each of the hash buckets}
\PY{n}{get\PYZus{}test\PYZus{}df} \PY{o}{=} \PY{l+s+s2}{\PYZdq{}\PYZdq{}\PYZdq{}}
\PY{l+s+s2}{SELECT}
\PY{l+s+s2}{	weight\PYZus{}pounds,}
\PY{l+s+s2}{    is\PYZus{}male,}
\PY{l+s+s2}{    mother\PYZus{}age,}
\PY{l+s+s2}{    plurality,}
\PY{l+s+s2}{    gestation\PYZus{}weeks,    }
\PY{l+s+s2}{FROM}
\PY{l+s+s2}{    (}\PY{l+s+si}{\PYZob{}CTE\PYZus{}first\PYZus{}bucketing\PYZcb{}}\PY{l+s+s2}{)}
\PY{l+s+s2}{WHERE}
\PY{l+s+s2}{     ABS(MOD(hash\PYZus{}values, }\PY{l+s+si}{\PYZob{}modulo\PYZus{}divisor\PYZcb{}}\PY{l+s+s2}{)) \PYZgt{}= }\PY{l+s+si}{\PYZob{}cum\PYZus{}eval\PYZus{}buckets\PYZcb{}}
\PY{l+s+s2}{     AND ABS(MOD(hash\PYZus{}values, }\PY{l+s+si}{\PYZob{}modulo\PYZus{}divisor\PYZcb{}}\PY{l+s+s2}{)) \PYZlt{} }\PY{l+s+si}{\PYZob{}modulo\PYZus{}divisor\PYZcb{}}
\PY{l+s+s2}{\PYZdq{}\PYZdq{}\PYZdq{}}\PY{o}{.}\PY{n}{format}\PY{p}{(}
    \PY{n}{CTE\PYZus{}first\PYZus{}bucketing}\PY{o}{=}\PY{n}{first\PYZus{}bucketing\PYZus{}query}\PY{p}{,} \PY{n}{modulo\PYZus{}divisor}\PY{o}{=}\PY{n}{modulo\PYZus{}divisor}\PY{p}{,}\PY{n}{cum\PYZus{}eval\PYZus{}buckets}\PY{o}{=}\PY{n}{train\PYZus{}buckets} \PY{o}{+} \PY{n}{eval\PYZus{}buckets}\PY{p}{)}
\end{Verbatim}
\end{tcolorbox}

    \textbf{Lab Task \#2:} Sample the natality dataset

    \begin{tcolorbox}[breakable, size=fbox, boxrule=1pt, pad at break*=1mm,colback=cellbackground, colframe=cellborder]
\prompt{In}{incolor}{43}{\boxspacing}
\begin{Verbatim}[commandchars=\\\{\}]
\PY{c+c1}{\PYZsh{} TODO 2}
\PY{c+c1}{\PYZsh{} TODO \PYZhy{}\PYZhy{} Your code here.}
\PY{c+c1}{\PYZsh{} every\PYZus{}n allows us to subsample from each of the hash values}
\PY{n}{train\PYZus{}df} \PY{o}{=} \PY{n}{dataframe\PYZus{}from\PYZus{}query}\PY{p}{(}\PY{n}{data\PYZus{}query}\PY{p}{,}\PY{l+m+mi}{100}\PY{p}{)}
\PY{n}{eval\PYZus{}df} \PY{o}{=} \PY{n}{dataframe\PYZus{}from\PYZus{}query}\PY{p}{(}\PY{n}{data\PYZus{}query}\PY{p}{,}\PY{l+m+mi}{100}\PY{p}{)}
\PY{n}{test\PYZus{}df} \PY{o}{=} \PY{n}{dataframe\PYZus{}from\PYZus{}query}\PY{p}{(}\PY{n}{data\PYZus{}query}\PY{p}{,}\PY{l+m+mi}{100}\PY{p}{)}
\PY{c+c1}{\PYZsh{} This helps us get approximately the record counts we want}
\PY{n+nb}{print}\PY{p}{(}\PY{l+s+s2}{\PYZdq{}}\PY{l+s+s2}{There are }\PY{l+s+si}{\PYZob{}\PYZcb{}}\PY{l+s+s2}{ examples in the train dataset.}\PY{l+s+s2}{\PYZdq{}}\PY{o}{.}\PY{n}{format}\PY{p}{(}\PY{n+nb}{len}\PY{p}{(}\PY{n}{train\PYZus{}df}\PY{p}{)}\PY{p}{)}\PY{p}{)}
\PY{n+nb}{print}\PY{p}{(}\PY{l+s+s2}{\PYZdq{}}\PY{l+s+s2}{There are }\PY{l+s+si}{\PYZob{}\PYZcb{}}\PY{l+s+s2}{ examples in the validation dataset.}\PY{l+s+s2}{\PYZdq{}}\PY{o}{.}\PY{n}{format}\PY{p}{(}\PY{n+nb}{len}\PY{p}{(}\PY{n}{eval\PYZus{}df}\PY{p}{)}\PY{p}{)}\PY{p}{)}
\PY{n+nb}{print}\PY{p}{(}\PY{l+s+s2}{\PYZdq{}}\PY{l+s+s2}{There are }\PY{l+s+si}{\PYZob{}\PYZcb{}}\PY{l+s+s2}{ examples in the test dataset.}\PY{l+s+s2}{\PYZdq{}}\PY{o}{.}\PY{n}{format}\PY{p}{(}\PY{n+nb}{len}\PY{p}{(}\PY{n}{test\PYZus{}df}\PY{p}{)}\PY{p}{)}\PY{p}{)}
\end{Verbatim}
\end{tcolorbox}

    \begin{Verbatim}[commandchars=\\\{\}]
There are 100 examples in the train dataset.
There are 100 examples in the validation dataset.
There are 100 examples in the test dataset.
    \end{Verbatim}

    \hypertarget{preprocess-data-using-pandas}{%
\subsection{Preprocess data using
Pandas}\label{preprocess-data-using-pandas}}

We'll perform a few preprocessing steps to the data in our dataset.
Let's add extra rows to simulate the lack of ultrasound. That is we'll
duplicate some rows and make the \texttt{is\_male} field be
\texttt{Unknown}. Also, if there is more than child we'll change the
\texttt{plurality} to \texttt{Multiple(2+)}. While we're at it, we'll
also change the plurality column to be a string. We'll perform these
operations below.

Let's start by examining the training dataset as is.

    \begin{tcolorbox}[breakable, size=fbox, boxrule=1pt, pad at break*=1mm,colback=cellbackground, colframe=cellborder]
\prompt{In}{incolor}{44}{\boxspacing}
\begin{Verbatim}[commandchars=\\\{\}]
\PY{n}{train\PYZus{}df}\PY{o}{.}\PY{n}{head}\PY{p}{(}\PY{p}{)}
\end{Verbatim}
\end{tcolorbox}

            \begin{tcolorbox}[breakable, size=fbox, boxrule=.5pt, pad at break*=1mm, opacityfill=0]
\prompt{Out}{outcolor}{44}{\boxspacing}
\begin{Verbatim}[commandchars=\\\{\}]
   weight\_pounds  is\_male  mother\_age  plurality  gestation\_weeks  \textbackslash{}
0       4.695846    False          31          1               32
1       5.687926     True          30          1               36
2       7.936641    False          24          1               39
3       8.198992     True          29          1               39
4       7.061406    False          40          1               37

           hash\_values
0 -8036809554133325397
1 -6132026233917866995
2  8439164539444335271
3   410994157116516864
4 -4947124986429933431
\end{Verbatim}
\end{tcolorbox}
        
    Also, notice that there are some very important numeric fields that are
missing in some rows (the count in Pandas doesn't count missing data)

    \begin{tcolorbox}[breakable, size=fbox, boxrule=1pt, pad at break*=1mm,colback=cellbackground, colframe=cellborder]
\prompt{In}{incolor}{45}{\boxspacing}
\begin{Verbatim}[commandchars=\\\{\}]
\PY{n}{train\PYZus{}df}\PY{o}{.}\PY{n}{describe}\PY{p}{(}\PY{p}{)}
\end{Verbatim}
\end{tcolorbox}

            \begin{tcolorbox}[breakable, size=fbox, boxrule=.5pt, pad at break*=1mm, opacityfill=0]
\prompt{Out}{outcolor}{45}{\boxspacing}
\begin{Verbatim}[commandchars=\\\{\}]
       weight\_pounds  mother\_age   plurality  gestation\_weeks   hash\_values
count     100.000000  100.000000  100.000000       100.000000  1.000000e+02
mean        7.431849   26.710000    1.030000        38.880000  5.249029e+17
std         1.200063    6.141916    0.171447         1.950291  5.345227e+18
min         4.499635   16.000000    1.000000        32.000000 -9.192824e+18
25\%         6.686620   21.000000    1.000000        38.000000 -3.573048e+18
50\%         7.593823   27.000000    1.000000        39.000000  7.224660e+17
75\%         8.190724   31.000000    1.000000        40.000000  5.152596e+18
max        10.500618   42.000000    2.000000        43.000000  9.221737e+18
\end{Verbatim}
\end{tcolorbox}
        
    It is always crucial to clean raw data before using in machine learning,
so we have a preprocessing step. We'll define a \texttt{preprocess}
function below. Note that the mother's age is an input to our model so
users will have to provide the mother's age; otherwise, our service
won't work. The features we use for our model were chosen because they
are such good predictors and because they are easy enough to collect.

    \textbf{Lab Task \#3:} Preprocess the data in Pandas dataframe

    \begin{tcolorbox}[breakable, size=fbox, boxrule=1pt, pad at break*=1mm,colback=cellbackground, colframe=cellborder]
\prompt{In}{incolor}{46}{\boxspacing}
\begin{Verbatim}[commandchars=\\\{\}]
   \PY{c+c1}{\PYZsh{} TODO 3}
   \PY{c+c1}{\PYZsh{} TODO \PYZhy{}\PYZhy{} Your code here.}
\PY{k}{def} \PY{n+nf}{preprocess} \PY{p}{(}\PY{n}{df}\PY{p}{)}\PY{p}{:}    
    \PY{c+c1}{\PYZsh{} Modify plurality field to be a string}
    \PY{n}{twins\PYZus{}etc} \PY{o}{=} \PY{n+nb}{dict}\PY{p}{(}\PY{n+nb}{zip}\PY{p}{(}\PY{p}{[}\PY{l+m+mi}{1}\PY{p}{,}\PY{l+m+mi}{2}\PY{p}{,}\PY{l+m+mi}{3}\PY{p}{,}\PY{l+m+mi}{4}\PY{p}{,}\PY{l+m+mi}{5}\PY{p}{]}\PY{p}{,}
                   \PY{p}{[}\PY{l+s+s2}{\PYZdq{}}\PY{l+s+s2}{Single(1)}\PY{l+s+s2}{\PYZdq{}}\PY{p}{,}
                    \PY{l+s+s2}{\PYZdq{}}\PY{l+s+s2}{Twins(2)}\PY{l+s+s2}{\PYZdq{}}\PY{p}{,}
                    \PY{l+s+s2}{\PYZdq{}}\PY{l+s+s2}{Triplets(3)}\PY{l+s+s2}{\PYZdq{}}\PY{p}{,}
                    \PY{l+s+s2}{\PYZdq{}}\PY{l+s+s2}{Quadruplets(4)}\PY{l+s+s2}{\PYZdq{}}\PY{p}{,}
                    \PY{l+s+s2}{\PYZdq{}}\PY{l+s+s2}{Quintuplets(5)}\PY{l+s+s2}{\PYZdq{}}\PY{p}{]}\PY{p}{)}\PY{p}{)}
    \PY{n}{df}\PY{p}{[}\PY{l+s+s2}{\PYZdq{}}\PY{l+s+s2}{plurality}\PY{l+s+s2}{\PYZdq{}}\PY{p}{]}\PY{o}{.}\PY{n}{replace}\PY{p}{(}\PY{n}{twins\PYZus{}etc}\PY{p}{,} \PY{n}{inplace}\PY{o}{=}\PY{k+kc}{True}\PY{p}{)}

    \PY{c+c1}{\PYZsh{} Clone data and mask certain columns to simulate lack of ultrasound}
    \PY{n}{no\PYZus{}ultrasound} \PY{o}{=} \PY{n}{df}\PY{o}{.}\PY{n}{copy}\PY{p}{(}\PY{n}{deep}\PY{o}{=}\PY{k+kc}{True}\PY{p}{)}

    \PY{c+c1}{\PYZsh{} Modify is\PYZus{}male}
    \PY{n}{no\PYZus{}ultrasound}\PY{p}{[}\PY{l+s+s2}{\PYZdq{}}\PY{l+s+s2}{is\PYZus{}male}\PY{l+s+s2}{\PYZdq{}}\PY{p}{]} \PY{o}{=} \PY{l+s+s2}{\PYZdq{}}\PY{l+s+s2}{Unknown}\PY{l+s+s2}{\PYZdq{}}
    
    \PY{c+c1}{\PYZsh{} Modify plurality}
    \PY{n}{condition} \PY{o}{=} \PY{n}{no\PYZus{}ultrasound}\PY{p}{[}\PY{l+s+s2}{\PYZdq{}}\PY{l+s+s2}{plurality}\PY{l+s+s2}{\PYZdq{}}\PY{p}{]} \PY{o}{!=} \PY{l+s+s2}{\PYZdq{}}\PY{l+s+s2}{Single(1)}\PY{l+s+s2}{\PYZdq{}}
    \PY{n}{no\PYZus{}ultrasound}\PY{o}{.}\PY{n}{loc}\PY{p}{[}\PY{n}{condition}\PY{p}{,} \PY{l+s+s2}{\PYZdq{}}\PY{l+s+s2}{plurality}\PY{l+s+s2}{\PYZdq{}}\PY{p}{]} \PY{o}{=} \PY{l+s+s2}{\PYZdq{}}\PY{l+s+s2}{Multiple(2+)}\PY{l+s+s2}{\PYZdq{}}

    \PY{c+c1}{\PYZsh{} Concatenate both datasets together and shuffle}
    \PY{k}{return} \PY{n}{pd}\PY{o}{.}\PY{n}{concat}\PY{p}{(}
        \PY{p}{[}\PY{n}{df}\PY{p}{,} \PY{n}{no\PYZus{}ultrasound}\PY{p}{]}\PY{p}{)}\PY{o}{.}\PY{n}{sample}\PY{p}{(}\PY{n}{frac}\PY{o}{=}\PY{l+m+mi}{1}\PY{p}{)}\PY{o}{.}\PY{n}{reset\PYZus{}index}\PY{p}{(}\PY{n}{drop}\PY{o}{=}\PY{k+kc}{True}\PY{p}{)}
\end{Verbatim}
\end{tcolorbox}

    Let's process the train, eval, test set and see a small sample of the
training data after our preprocessing:

    \begin{tcolorbox}[breakable, size=fbox, boxrule=1pt, pad at break*=1mm,colback=cellbackground, colframe=cellborder]
\prompt{In}{incolor}{47}{\boxspacing}
\begin{Verbatim}[commandchars=\\\{\}]
\PY{n}{train\PYZus{}df} \PY{o}{=} \PY{n}{preprocess}\PY{p}{(}\PY{n}{train\PYZus{}df}\PY{p}{)}
\PY{n}{eval\PYZus{}df} \PY{o}{=} \PY{n}{preprocess}\PY{p}{(}\PY{n}{eval\PYZus{}df}\PY{p}{)}
\PY{n}{test\PYZus{}df} \PY{o}{=} \PY{n}{preprocess}\PY{p}{(}\PY{n}{test\PYZus{}df}\PY{p}{)}
\end{Verbatim}
\end{tcolorbox}

    \begin{tcolorbox}[breakable, size=fbox, boxrule=1pt, pad at break*=1mm,colback=cellbackground, colframe=cellborder]
\prompt{In}{incolor}{48}{\boxspacing}
\begin{Verbatim}[commandchars=\\\{\}]
\PY{n}{train\PYZus{}df}\PY{o}{.}\PY{n}{head}\PY{p}{(}\PY{p}{)}
\end{Verbatim}
\end{tcolorbox}

            \begin{tcolorbox}[breakable, size=fbox, boxrule=.5pt, pad at break*=1mm, opacityfill=0]
\prompt{Out}{outcolor}{48}{\boxspacing}
\begin{Verbatim}[commandchars=\\\{\}]
   weight\_pounds  is\_male  mother\_age  plurality  gestation\_weeks  \textbackslash{}
0       8.688418     True          31  Single(1)               41
1       6.393406  Unknown          30  Single(1)               38
2       5.136771  Unknown          21  Single(1)               37
3       6.188376  Unknown          42  Single(1)               37
4       7.625790  Unknown          22  Single(1)               38

           hash\_values
0 -1569657028734518022
1 -1052589534453650062
2 -7363173917873728029
3  7782266297148452291
4  7579041105174423352
\end{Verbatim}
\end{tcolorbox}
        
    \begin{tcolorbox}[breakable, size=fbox, boxrule=1pt, pad at break*=1mm,colback=cellbackground, colframe=cellborder]
\prompt{In}{incolor}{49}{\boxspacing}
\begin{Verbatim}[commandchars=\\\{\}]
\PY{n}{train\PYZus{}df}\PY{o}{.}\PY{n}{tail}\PY{p}{(}\PY{p}{)}
\end{Verbatim}
\end{tcolorbox}

            \begin{tcolorbox}[breakable, size=fbox, boxrule=.5pt, pad at break*=1mm, opacityfill=0]
\prompt{Out}{outcolor}{49}{\boxspacing}
\begin{Verbatim}[commandchars=\\\{\}]
     weight\_pounds  is\_male  mother\_age  plurality  gestation\_weeks  \textbackslash{}
195       8.198992  Unknown          29  Single(1)               39
196       7.125340    False          25  Single(1)               40
197       7.625790     True          26  Single(1)               42
198       6.492614  Unknown          28  Single(1)               41
199       8.785421  Unknown          24  Single(1)               41

             hash\_values
195   410994157116516864
196  3190725018108452514
197  5655500751972499290
198  8297008456970080747
199  5617905498255901254
\end{Verbatim}
\end{tcolorbox}
        
    Let's look again at a summary of the dataset. Note that we only see
numeric columns, so \texttt{plurality} does not show up.

    \begin{tcolorbox}[breakable, size=fbox, boxrule=1pt, pad at break*=1mm,colback=cellbackground, colframe=cellborder]
\prompt{In}{incolor}{50}{\boxspacing}
\begin{Verbatim}[commandchars=\\\{\}]
\PY{n}{train\PYZus{}df}\PY{o}{.}\PY{n}{describe}\PY{p}{(}\PY{p}{)}
\end{Verbatim}
\end{tcolorbox}

            \begin{tcolorbox}[breakable, size=fbox, boxrule=.5pt, pad at break*=1mm, opacityfill=0]
\prompt{Out}{outcolor}{50}{\boxspacing}
\begin{Verbatim}[commandchars=\\\{\}]
       weight\_pounds  mother\_age  gestation\_weeks   hash\_values
count     200.000000  200.000000       200.000000  2.000000e+02
mean        7.431849   26.710000        38.880000  5.249029e+17
std         1.197044    6.126465         1.945385  5.331780e+18
min         4.499635   16.000000        32.000000 -9.192824e+18
25\%         6.686620   21.000000        38.000000 -3.573048e+18
50\%         7.593823   27.000000        39.000000  7.224660e+17
75\%         8.190724   31.000000        40.000000  5.152596e+18
max        10.500618   42.000000        43.000000  9.221737e+18
\end{Verbatim}
\end{tcolorbox}
        
    \hypertarget{write-to-.csv-files}{%
\subsection{Write to .csv files}\label{write-to-.csv-files}}

In the final versions, we want to read from files, not Pandas
dataframes. So, we write the Pandas dataframes out as csv files. Using
csv files gives us the advantage of shuffling during read. This is
important for distributed training because some workers might be slower
than others, and shuffling the data helps prevent the same data from
being assigned to the slow workers.

    \begin{tcolorbox}[breakable, size=fbox, boxrule=1pt, pad at break*=1mm,colback=cellbackground, colframe=cellborder]
\prompt{In}{incolor}{51}{\boxspacing}
\begin{Verbatim}[commandchars=\\\{\}]
\PY{c+c1}{\PYZsh{} Define columns}
\PY{n}{columns} \PY{o}{=} \PY{p}{[}\PY{l+s+s2}{\PYZdq{}}\PY{l+s+s2}{weight\PYZus{}pounds}\PY{l+s+s2}{\PYZdq{}}\PY{p}{,}
           \PY{l+s+s2}{\PYZdq{}}\PY{l+s+s2}{is\PYZus{}male}\PY{l+s+s2}{\PYZdq{}}\PY{p}{,}
           \PY{l+s+s2}{\PYZdq{}}\PY{l+s+s2}{mother\PYZus{}age}\PY{l+s+s2}{\PYZdq{}}\PY{p}{,}
           \PY{l+s+s2}{\PYZdq{}}\PY{l+s+s2}{plurality}\PY{l+s+s2}{\PYZdq{}}\PY{p}{,}
           \PY{l+s+s2}{\PYZdq{}}\PY{l+s+s2}{gestation\PYZus{}weeks}\PY{l+s+s2}{\PYZdq{}}\PY{p}{]}

\PY{c+c1}{\PYZsh{} Write out CSV files}
\PY{n}{train\PYZus{}df}\PY{o}{.}\PY{n}{to\PYZus{}csv}\PY{p}{(}
    \PY{n}{path\PYZus{}or\PYZus{}buf}\PY{o}{=}\PY{l+s+s2}{\PYZdq{}}\PY{l+s+s2}{train.csv}\PY{l+s+s2}{\PYZdq{}}\PY{p}{,} \PY{n}{columns}\PY{o}{=}\PY{n}{columns}\PY{p}{,} \PY{n}{header}\PY{o}{=}\PY{k+kc}{False}\PY{p}{,} \PY{n}{index}\PY{o}{=}\PY{k+kc}{False}\PY{p}{)}
\PY{n}{eval\PYZus{}df}\PY{o}{.}\PY{n}{to\PYZus{}csv}\PY{p}{(}
    \PY{n}{path\PYZus{}or\PYZus{}buf}\PY{o}{=}\PY{l+s+s2}{\PYZdq{}}\PY{l+s+s2}{eval.csv}\PY{l+s+s2}{\PYZdq{}}\PY{p}{,} \PY{n}{columns}\PY{o}{=}\PY{n}{columns}\PY{p}{,} \PY{n}{header}\PY{o}{=}\PY{k+kc}{False}\PY{p}{,} \PY{n}{index}\PY{o}{=}\PY{k+kc}{False}\PY{p}{)}
\PY{n}{test\PYZus{}df}\PY{o}{.}\PY{n}{to\PYZus{}csv}\PY{p}{(}
    \PY{n}{path\PYZus{}or\PYZus{}buf}\PY{o}{=}\PY{l+s+s2}{\PYZdq{}}\PY{l+s+s2}{test.csv}\PY{l+s+s2}{\PYZdq{}}\PY{p}{,} \PY{n}{columns}\PY{o}{=}\PY{n}{columns}\PY{p}{,} \PY{n}{header}\PY{o}{=}\PY{k+kc}{False}\PY{p}{,} \PY{n}{index}\PY{o}{=}\PY{k+kc}{False}\PY{p}{)}
\end{Verbatim}
\end{tcolorbox}

    \begin{tcolorbox}[breakable, size=fbox, boxrule=1pt, pad at break*=1mm,colback=cellbackground, colframe=cellborder]
\prompt{In}{incolor}{52}{\boxspacing}
\begin{Verbatim}[commandchars=\\\{\}]
\PY{o}{\PYZpc{}\PYZpc{}}\PY{k}{bash}
wc \PYZhy{}l *.csv
\end{Verbatim}
\end{tcolorbox}

    \begin{Verbatim}[commandchars=\\\{\}]
  200 eval.csv
  200 test.csv
  200 train.csv
  600 total
    \end{Verbatim}

    \begin{tcolorbox}[breakable, size=fbox, boxrule=1pt, pad at break*=1mm,colback=cellbackground, colframe=cellborder]
\prompt{In}{incolor}{53}{\boxspacing}
\begin{Verbatim}[commandchars=\\\{\}]
\PY{o}{\PYZpc{}\PYZpc{}}\PY{k}{bash}
head *.csv
\end{Verbatim}
\end{tcolorbox}

    \begin{Verbatim}[commandchars=\\\{\}]
==> eval.csv <==
7.5618555866,False,20,Single(1),43
4.7509617461,False,36,Single(1),39
6.75055446244,True,21,Single(1),38
7.5618555866,Unknown,20,Single(1),43
7.91239058318,Unknown,30,Single(1),39
5.4454178714,Unknown,30,Single(1),35
7.06140625186,Unknown,40,Single(1),37
6.87401332916,True,29,Single(1),38
8.377565956,Unknown,28,Single(1),42
7.5618555866,Unknown,20,Single(1),40

==> test.csv <==
7.12534030784,True,18,Single(1),42
7.0988848364,Unknown,28,Single(1),40
9.56365292556,Unknown,30,Single(1),40
9.06320359082,Unknown,29,Single(1),41
6.1883756943399995,True,17,Single(1),36
4.7509617461,Unknown,36,Single(1),39
6.41324720158,Unknown,29,Single(1),38
6.87621795178,Unknown,19,Single(1),38
5.1367707046,True,21,Single(1),37
9.16902547658,Unknown,19,Single(1),40

==> train.csv <==
8.68841774542,True,31,Single(1),41
6.393405598,Unknown,30,Single(1),38
5.1367707046,Unknown,21,Single(1),37
6.1883756943399995,Unknown,42,Single(1),37
7.62578964258,Unknown,22,Single(1),38
8.344496616699999,False,22,Single(1),39
7.81318256528,Unknown,17,Single(1),37
7.936641432,True,36,Single(1),38
8.93754010148,Unknown,31,Single(1),40
8.3665428429,True,25,Single(1),40
    \end{Verbatim}

    \begin{tcolorbox}[breakable, size=fbox, boxrule=1pt, pad at break*=1mm,colback=cellbackground, colframe=cellborder]
\prompt{In}{incolor}{54}{\boxspacing}
\begin{Verbatim}[commandchars=\\\{\}]
\PY{o}{\PYZpc{}\PYZpc{}}\PY{k}{bash}
tail *.csv
\end{Verbatim}
\end{tcolorbox}

    \begin{Verbatim}[commandchars=\\\{\}]
==> eval.csv <==
6.393405598,Unknown,30,Single(1),38
6.944561253,Unknown,32,Single(1),37
6.4992274837599995,True,31,Twins(2),37
7.5618555866,True,20,Single(1),40
7.25100379718,Unknown,28,Single(1),39
5.3241636273,True,38,Twins(2),34
6.0406659788,Unknown,33,Single(1),37
7.12534030784,True,18,Single(1),42
6.4926136159,Unknown,28,Single(1),41
7.12534030784,Unknown,25,Single(1),39

==> test.csv <==
5.8753192823,True,21,Single(1),36
7.13856804356,Unknown,22,Single(1),38
7.12534030784,Unknown,18,Single(1),42
5.4454178714,Unknown,30,Single(1),35
7.06140625186,Unknown,40,Single(1),37
7.31273323054,Unknown,23,Single(1),39
5.8135898489399995,False,37,Single(1),37
9.93843877096,Unknown,28,Single(1),41
7.936641432,False,19,Single(1),40
7.0988848364,False,28,Single(1),40

==> train.csv <==
8.375361333379999,True,30,Single(1),40
7.3744626639,Unknown,26,Single(1),39
4.7509617461,False,36,Single(1),39
6.75055446244,Unknown,29,Single(1),38
7.91239058318,False,30,Single(1),39
8.19899152378,Unknown,29,Single(1),39
7.12534030784,False,25,Single(1),40
7.62578964258,True,26,Single(1),42
6.4926136159,Unknown,28,Single(1),41
8.7854211407,Unknown,24,Single(1),41
    \end{Verbatim}

    \hypertarget{lab-summary}{%
\subsection{Lab Summary:}\label{lab-summary}}

In this lab, we set up the environment, sampled the natality dataset to
create train, eval, test splits, and preprocessed the data in a Pandas
dataframe.

    Copyright 2020 Google Inc.~Licensed under the Apache License, Version
2.0 (the ``License''); you may not use this file except in compliance
with the License. You may obtain a copy of the License at
http://www.apache.org/licenses/LICENSE-2.0 Unless required by applicable
law or agreed to in writing, software distributed under the License is
distributed on an ``AS IS'' BASIS, WITHOUT WARRANTIES OR CONDITIONS OF
ANY KIND, either express or implied. See the License for the specific
language governing permissions and limitations under the License


    % Add a bibliography block to the postdoc
    
    
    
\end{document}
